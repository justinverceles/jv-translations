\documentclass[12pt,a4paper]{article}
\usepackage[english]{babel}
\usepackage{microtype}
\usepackage[a4paper,margin=2in]{geometry}
\usepackage[T1]{fontenc}
\usepackage{fontspec}
\usepackage{ebgaramond}
\tolerance=1385

\title{The Visage\\ \large{\textit{A half-hour in a mind}}}
\author{by Richard Dehmel\\translated from the German by Justin Verceles}
\date{}

\begin{document}

\maketitle
He sat and couldn’t free himself from this oppressive spell. It lowered and lowered over him like a magnetic ring around his brow and disabled his hand. For weeks already—since he’d recovered. Always when he wanted to paint. Always the one, great, unfulfilled desire—the object of his hundred merry struggles and sketches: the painting, the painting—her face! however much he wished to begin something new.

He heard her through the carpet, busying about in the next room. It sounded so strange and subdued. And the burns on the carpet—how painfully they reminded him! He felt his strong shoulders twitch inavertibly. He regarded with contempt and exhaustion the landscape on the canvas, threw his brush away, and looked shyly up to the wall, to that image of man.

There it hung and waited, the last of many; she’d salvaged it from the fire at the last moment, from the flying flames. It was like an incubus—this unsettled assignment, this face.

Oh certainly, it was finished—it was a painting, a painting as only \textit{he} could produce: this lady, there with the narcissus in her severely folded hands. You could almost smell it, the shining, immaculate blossom, bent forward with the little purple and yellow crown upon the white star; the intoxicating blossom before her young, full, naked breasts. And thereover her mutely permitting mouth. And still above that her menacing blue eyes, large and dark, directed into the distance. And finally thereover all the brilliance of her hair, heavy, red, and gilt like copper-gold, beshaded dark-green by the old myrtle tree’s thick foliage, with the small bright white swelling buds. Indeed, his friends had chided that he didn’t want the world to see it; that was then.

But this was it exactly—they shouldn’t see it now either! Never, never until he’d found the one thing it still wanted, visible only to \textit{him}—what only \textit{he} found lacking in these pictures, the final riddle of her visage: why he loved her.

Oh, and now it was impossible—it was destroyed, this silent living riddle; the mystery of her features, consumed by the flames; that stately neck, tattered with scars; those supple lips—and all for his sake! And he’d known it, known with his whole being that he would finally succeed in discovering it and putting it to canvas, this allective wonder. Not in the eyes, nor in the corners of her mouth. It didn’t lie there or in any one detail. And not in the mood either; he’d tried and come up against all that. It was an expression, an expression! And he’d gotten so close in his last painting, which hung up there on the wall—the sole remnant. And now—now? He pressed his fingers together; he would’ve liked to squeeze them bloody.

And all because he loved her—precisely because of that. And because he was so strong. But suppose there were ramifications? Ramifications to strength? In itself? Was that why he’d broken his foot?

Or suppose love was a sin? Not altogether, but for \textit{him}—a sin against art! Overwhelming! For it hadn’t been like this straight away; what should he care about her soul? But gradually—oh, but that was just it. Holy for even the artist, that was what opened his eyes, the absolute holiest thing about the form: the spellbinding soul, the reciprocity of all living things!

And thus it eventuated: of model and woman, of body and being, and increasing reciprocity of the artist and her beauty, and still of man and her sex. No, he didn’t want that. He only wanted her with his eyes—her eyes, the floating flowers, dark and blue as night; her clear-featured face, placid as a forest lake; everything! And yet—how he then discerned her, this figure, look for look, and clue by clue grew surer, faster in the picture, and how everything in his senses stretched after her, how her ardency swelled with his longing; aye, it was nature, nature! Was this impotence?

That moment, after that last painting, when he grabbed her wrist, still trembling for productive enchantment, and showed her that new expression which nearly unriddled her: this yearnful chastity; and then she gazed with burning thirst, and surveyed the last painting, so that she couldn’t bear it anymore and stumbled down to him, so warm and weighty, and he to her—oh, engrossment! And then, then—it was too hard, too absurdly hard of fate, how he snatched her up with wild arms, crying from desire and redoubled happiness, and leaped with her over the footstool—this wicked ankle fracture; he could still laugh about it then, in his revelous love—that was then.

He listened. What might she be thinking of now? Him alone. That he felt. There lay the difficulty—the magnetic ring.

How quietly she sat back down. So he would better not perceive her, there in the little chamber, behind the carpet—nothing stirred; so it remained day after day. And in the evenings arose the unease, the furtive unease wherewith she kept in the darkness, in the gloom, or enshrouded her face, so he would better not see her—so he would better just forget her, her dead beauty, her soul’s image, this harrowing impossibility. Yes, the unease in the air, that was it; it obliterated him, this kind of love.

And was this indeed love? This disabling pressure! Wasn’t it all but memory?!

Even nights were insecure—no sooner could he touch her before it appeared before him again, the whole horrific ruddy drama, and robbed him his senses, hot and cold. How she’d awakened him, lifted him and his sick, thickly splinted-up foot from the smoking bed, the flames already licking at her back, through the door and down the twelve dark steps—oh, she was strong, almost as strong as he!—and then rushed back, notwithstanding protest, to save the painting, at least the one, back up into the glowing quadrate, with her long open braids, which flowed like rolling waves in the firelight—that flicker! And suddenly the cry, that shrill, splitting cry, and the painting clattering down to him; and she above, large, in dreadful splendor, with her snatching arms, her red hairs disintegrating into bluish sparks, a scintillant halo! Fluttering wings about her heaving bosom! And the gruesome, glinting eyes! And \textit{he}, writhing helplessly down below! And once more that cry, that fervent, bestial cry! And his own cry: how she turned back around as a single flaming sheaf and returned once more—so that he lost his senses—until they awakened him and she lay beside him, wrapped in the carpet, after she’d run back in a final moment of terrible prudence to smother her blazing torment, the strong, valiant creature—his savior!

Perhaps he could paint that—fiery wings? No, nonsense; he might as well paint the sunbeam glittering on his palette. Ah, sunlight! How her hair used to shimmer in it, so smooth and wavy; suppose it would grow back? But what good would that do! Her face, that was the irreparable part! That was the memory that drew him to—no—away from her.

He stared at the floor. If she’d died—really died, not just in him—then he could pray to her all his life, in sad silence, like a child to the Virgin Mary. No, it’d always been Mary Magdalene; inwardly he’d always thought of her since he’d had to kneel and pray as punishment and had secretly bought himself the Bible. Magdalene, the loving sinner.

Oh, this accursed brooding. She was alive, alive and loved him—and healthy, healthy as he. Oh, that beautiful, flourishing word! Oh, her tortuous ugliness! Her remindful presence! Desire and revulsion! Impotence!

He looked back up—to the carpet, to the narcissus. What if he sold it? Maybe then he’d have peace. Of what use was this senselessly obsessing over a single bit of soul? Of what use was the whole pedantic melancholy? Why couldn’t he content himself with a colorful comicality as others did—with the glittery displays he otherwise derided? The solution was simple after all: try something new! But still she would remain. And should he cut the painting up into pieces, the memory would remain as long as she remained—and with her the pressure. And he couldn’t paint the memory.

Freedom! Aye—that was the unhealthy thing, it was immoral, this dull, unnatural companionship! This thraldom! This bondage!

He gazed at his palette; a cloud’s shadow had wiped out the light beam. What if he gave her makeup? Disgusting! And still the form remained ruined, the soul in the visage. And her shame! Her pride! Then she would go!

But was that what he wanted? Then put the painting on exhibition—hence with it! A tour—arctic sun! Only one, two years would suffice to pay for the painting and the rest of his inheritance; he would just work. He’d learned enough from her! He wanted to show others already why he’d sat in stillness for so long.

And she? She was clever enough, the professor’s daughter. She could give lessons or become a bookkeeper; or he would send her something himself. No, disgraceful—she wouldn’t take that. And—and if people didn’t want her? With her deformed face?!

Oh, conscience! Why did he have this conscience! Indeed—for art, it was good. But for life? For life it was needless! Not because he’d seduced her—no! She’d rather seduced him. Or because her family ostracized her? An outcast?! And for his sake! No—that’d been her own fault. Why then had she returned before he’d had even an inkling of love? And over and over until she had to stay. That was her own undoing—her will!

Because his earnestness attracted her—say her parents what they might. Because she felt his pure will. But was it pure? Yes! Until he lost it, in that moment, his will to the form. No, before that: until he saw the soul. But that was the form, the spellbinding soul; that was what he’d sought, what she’d sensed, why she trusted him—him, the artist. No, and the man! The man who stood above himself, above nature, above life and soul, perforce of his form-mastering spirit! And yet not so! It was the same nature, the same senses, the same spirit: the force of the artist, of man.

Yes, there was the nub: that moment, that image; his art, his life; his will, her will—it was all the same menacing, anguishing thing! For he owed her his life—her, his savior! His life, his art, his soul—his whole trade and purpose in the world.

He started—the shadow of a new cloud crept through the stillness. He squeezed his eyes shut. He didn’t want to see it anymore, the demanding, menacing picture; he hated it. He pressed his fists against his eyes so his vision shimmered. It only appeared mightier in the sparkling sheen; and he saw her, saw her as she was now, with her stiff, shapeless mouth, with her hairless head, with the scars about her cheeks and chin, with her bare, red-welted neck. He groaned loudly, so that he frightened from the hollow, lonely voice.

Hear—surely it wasn’t his voice? Then hesitantly it came searching through the great room: “Did you call?” Soft and weighty as the carpet, which he heard shift around.

He didn’t look up. He felt her standing there with an inquiring face—not her face! He wanted to speak. Then she came.

He wanted to shake his head—but her hand on his shoulder, her expectation! It was impossible, it hoisted him to his feet. He had to look at her, past her gray morning dress—her neck! And—the red! The whithering black! The soul! The gaze! Her face! That was the fiercest part—there she stood, high, stiff, trembling: “I’ll go,” and she began to turn around.

And \textit{he}—he looked at her, and his eyes widened and widened, so that she couldn’t free herself—looking more and more—and his fingers reached and felt, to grasp it, to hold it: that last one; that unknown; that divine wonder; what drove him to his knees, to embracing her, weeping, stammering, “Revelation”—her great morality, the beauty of her upheaval!

And now, softly—softly, weightily, silently—she sank down to him, knee to knee, with childlike piety, different from then. And he kissed her shapeless lips, placed his hands around her hairless head, and held her before him, looking, looking. No, it didn’t lie in her eyes, nor in the corners of her mouth, nor in any particular detail—should she lie before him perfectly enshrouded, it would bring him to reverence: this sublime highness, this blessed, prevailing humility.

And he laughed and had to say the superfluous: “I love you.”

And when, in their clarity, they rose from their knees, and the broad sunbeam glittered on the palette—it ascended there before him, new and mighty, over to the wall, to the myrtle: “Do you know how I’m going to paint you? Storm and night; torch-flame; but movement and regard: the Magdalene ecstatically bearing the crucified!”

“Away from the cross,” spoke her soul.
\end{document}