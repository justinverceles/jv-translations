\documentclass[12pt,a4paper]{article}
\usepackage[english]{babel}
\usepackage{microtype}
\usepackage[a4paper,margin=2in]{geometry}
\usepackage[T1]{fontenc}
\usepackage{fontspec}
\usepackage{ebgaramond}
\tolerance=4820

\title{The Three Sisters\\ \large{\textit{A narrative with an audience}}}
\author{by Richard Dehmel\\translated from the German by Justin Verceles}
\date{}

\begin{document}

\maketitle

“All right—so—the story I want to tell you today—,” the amtmann began deliberately, “its undertaking befits only a proper storywriter if it’s ever to become something. And since my own knowledge of it comes mostly from mere bits and pieces I’ve gathered listening to others, I’ll have to take great effort to wheel it along. But I tell you—”

“My, you’ve got some shot in your chamber,” growled the forester; “Just shoot!” The town bookseller elbowed his friend’s side, gave a sly look, and said haughtily, “Trying to create tension, Furchenrat? Nice trick, my friend! Nothing new, nothing doing—however much you embellish.”

“Embellishment won’t be necessary this time,” the amtmann defended himself. “Though the story is simple enough,” he added gravely after a while. “So—”

“Just a moment! Top-ups first.” The innkeeper hefted himself from his upholstered armchair and filled the glasses afresh. Then he closed the little back room’s doors and rejoined their table. The amtmann had reclined in the old, high sofa, lost in thought like a cricket in a ditch. The room was completely still; one could hear the smoke-filled hanging lamp’s flame fret its wick. The gentlemen remarked vivid memories burning in the gaunt little man, and his mood diffused among them.

“So”—he erected himself halfway and stroked his sparse hair—“Yes! Very simple.” His threadbare voice nearly tore. “The girl”—he ran his hand over his skull once again and straightened up—“Right! The first time I saw her, I felt as though I’d known her all my life. And later I realized why: the way she walked, moved, and was—it was like what you read in fairy tales.

“Back then I was still as green as a grasshopper and about two years older than she. I’d only just come of age and was meant to begin work on my father’s estate after I’d seen enough of the world and served my year in the military, during which time the old man had taken her on.

“No one quite knew whence she’d originated, but our people always said she’d come \textit{vom leewen Godd}: from the Good Lord.

“And so it had come to pass that one day, she knocked on our door and requested a position as a maid. My father, who otherwise scarcely repeats himself, especially to us children, has told me twice how she’d regarded him so movingly yet so assuredly that he couldn’t refuse her, though he felt he had almost too many hands in his employ as it was. The next day, the priest from a neighboring Catholic village sent her meager belongings, and now and then he inquired whether she was being a good, able help. In those early days, she was the talk of the town, though, unusually, the talk was never sharp and derisive but always distinctly cautious and serious to the point of piety, so finally she was dubbed ‘Little Ladybird’; as it happened, her name was Marie. But the old man kept all her circumstances quiet. And it wasn’t easy to ask him about anything he didn’t discuss himself.

“And soon he felt there are some helps you can never have enough of. And because she bustled so deftly about the house, stable, and barn and because everything she handled enjoyed the same dominical cleanliness, he made her head maid after just a year, and none of the other girls begrudged her it either. They were at her beck and call without her even needing to command them, so profoundly did the gentle serenity of her countenances and the silent conviction of her conduct affect these simple souls; she couldn’t be called voluble. Yet there was nothing hidden in her, nothing others were left lost for words about. You could see her thoughts play in her eyes like fish in a clear pond, and whomever she paid her amiable attention made as frivolous use of their words as usual and, after leaving, believed to have heard a mouthful. But when her turn came to speak, you could confidently weigh her every word. And whatever she did, she did with both arms and standing on both legs.”

The bookseller had coughed twice already and blinked hard. But now the fulsome praise might have gotten too sweet for him, for he burst out with, “So you lost your heart to her!” The forester smirked and the tips of his whiskers bristled. The innkeeper gently caressed his ears like someone whose eardrums have long rattled from the beat of human things.

The amtmann realized he’d revealed more than he wanted to. But as a man who has finished reckoning with himself and can now speak calmly of life, he gave the measured reply: “Yes, gentlemen. I haven’t quite taken to another woman since her.”

Now the others too became serious again, and the bookseller regarded his friend’s gray hairs almost reverently—his lonesome, lovelorn friend, so temperamentally reserved yet so glad to tell his story, the key to his inmost core having now slipped from his careful fingers.

“Right, so”—he began anew—“by then I’d untied many an apron’s bow, for the words have always sprung from my lips like ripe peas from a pod. Which pleased the young maidens. But this one—she was the only one for whom I couldn’t poetize, such a yen did I have for her. But so did everyone. For she had a weakness.

“You see, old chapter-chapman,” he said, turning to his friend, “now’s the time to enumerate her flaws—or rather, her flaw. Yes, just one flaw to keep her all but lavishly equipped for life: her self-consciousness. You couldn’t talk to her about herself without her taking fright like a snail whose tentacles you’d tried touching. And long after it’d happened, you could still read the fear in her face whenever you came near her. So we all took care not to disturb her peaceful activity.

“So lay certainty and timidity, prudence and diffidence side-by-side in her soul frailly like uninterwoven threads. But this especially touched me, as perhaps it did my father, since my late mother had been similarly disposed albeit more splendorous and less delicate of figure.

“On our estate, there was only one man whose remarks she never shrunk away from, no matter what: Heinrich Wendel. Of course, this came about only very gradually; usually his words crept e’er-so-warily from his mouth like rainworms from the earth, hence his nickname, ‘Silent Heinz,’ or more often—in our northern tongue—\textit{Drömling}: Dreamling. He had a good year on me, and we knew each other since childhood. We’d sat together at the capital’s university—the gymnasium, as we call it. Only in the lower grades, though. By the fifth year, he’d gotten held back and left school to focus on nurturing the worms under his towhead.

“That is to say he wasn’t actually dumb; he just always had more in his head than was immediately relevant. And when you asked him a question, in his eyes dawned that dazzled smile of a child awakened from a dream. But because he was a handsome sort, most teachers liked having him and carried him along some more years, especially since occasionally he produced good work. Lads certainly teased him for his adrift disposition, so gradually he became more taciturn and placid-minded than was his nature; except that now and then, an astonishing exuberance would suddenly inspire him and strangely impel the entire class to childish frisks and dances.

“Such behavior overcame him on his confirmation day—already fifteen then—because he was allowed to return home and nestle up in his property, which lay about two miles from ours. He lived there in the company of his three sisters, who were all older than he and pampered him might and main. The mother, a plain, contemplative woman, had died years before—of grief for her husband, they said.

“Namely, his father was one of those modern, well-landed farmers whose leisurely avarice of yore had been pickpocketed, so to speak, in our present-day helter-skelter and who amassed their fortunes on the city banks’ railways rather than behind a field’s plow. He’d speculated in American mining securities and been exceptionally lucky. Yet meanwhile the hedonism that closely follows small successes into city life enkindled his fervid blood, and one learned that his way home found mostly wine taverns and still baser diversions. And this lifestyle proved his undoing. For one evening in the village inn, when he was pronounced a swindler, an inhuman fury overpowered him in his intoxicated rage and he suffered a stroke that killed him the next day.”

Now the innkeeper too divested himself of a stirring approval for his guest’s story and went, “Hm!” to which he also seemingly wanted to append an explanation, for his cheeks rounded as though he were rolling a word on his tongue. But the amtmann wasn’t to be disturbed. He continued:

“Nonetheless he’d been old-fashioned or wise enough or yielded the ghost timely enough to not squander his fortunes in more ventures. Rather he’d put most of it into land and amassed a sizable amount of pasture and country around his farmstead. And the estate he bequeathed to his children was the finest farm in the region and rivaled many great ancestral manors.

“So after the old man’s death, the four siblings ensconced themselves there. For while he was alive, he hadn’t suffered to have them around—partly to be left less disturbed within his scattered mind and partly due to possible shame over his untamed but healthy heart before his growing daughters. But most of all, they were to ‘learn to go along with the world.’ Which everyone construes differently. And while he prided himself amply in his dung-fork beginnings and, as often as not, would boast his ineducation and hard hands, his children were to know decency and a noble lifestyle; maybe he even dreamed of aristocratic sons-in-law.

“The young ladies, however, seemed of the opinion that the wisdom and worldly comprehension their father extolled could be found abundantly enough in his riches, for when at long last they returned home from the world, they retained little more than the modish sundries on their persons; it came as no surprise when the good old-line families they sought to crowd soon made themselves clear with some varnished disparagements, deterring further visits.

“Naturally this didn’t hurt their pride in the least; rather, they smugly complained about others’ pride or vulgarity and for a time found great edification in such comforting talk. What remarkable harmony in which they lived together. Like rats with enough to eat. And they were distinguishable only by age. Otherwise not much more could be said of them except that they were lucky to live—and that others were unlucky.”

The storyteller took thought; he shouldn’t reveal too much too early. “But,” he veered, “man’s will is to activity and to make himself useful, however he may. So slowly they wearied of belittlement, especially with no one  there to contradict them. Finally they conceived the plan to attend to their brother’s upbringing, for his will was solely to indolence. In the summer he’d loll at length in the bushes or at the field margins, watching the birds and beetles, while winters he spent sat in spinning rooms, listening to the old women’s ghost stories and other horrific tales. Only now and then he’d sit sedulously bent over thrilling adventure novels whose contents he’d later relate even more thrillingly.”

Here the bookseller raised his eyebrows and wanted to interject something. But his protest would remain snagged on his forehead wrinkles, for the amtmann obstinately kept on as though he were reading one of his transcripts.

“Quite regrettably, the military too wouldn’t take him when at muster his chest was adjudged too weak; otherwise maybe they would’ve bent him somewhat into shape. So only the sisters seemed to have received a piece each of the old man’s acuity and busyness. And though they had nothing against Heinz so squandering his time when he could afford it, they thought that the character of his proclivities besmirched him, and they probably would’ve rather seen him amble along his father’s sordid ways. Thus they made a point of gradually awakening him to the great dignity of his future landownership.

“Certainly this was tedious work, though they had no shortage of time or even patience; as I said, they really did love their brother in their manner, and the little feeling and faculty they had were engaged in worrying about him, or rather in the nestlecock they’d made of him. Hence they were all the keener to mold him to their sensibility. And truly Heinz wasn’t someone who gave real cause for ire or admonition. Rather, most people liked him and his instinctive pliability—he did and allowed anything asked of him, almost like a sleepwalker—which quality was also the only fault repeatedly found with him.

“And all that their insistence determined was that a certain easy mulishness increasingly subjugated his disjointed mind. For he’d always minded others’ doings far too little to adequately play the lord his sisters envisaged, and he didn’t esteem their cultivation enough to be thus convertible to their wisdom.”

The word “cultivation” redisturbed the bookseller, who shook his high cocked-up head most displeasedly when the amtmann ignored him once again and said hastily, “But to really fulfill the lordly role in his manor, he needed, above all, desire and experience. He himself discerned his lack thereof and shied from interrupting his workers in their affairs.

“Supposedly even his distraitness once left him headlong and he went wild and threatened the eldest sister with a knife after she’d pressed him too strongly to punish a servant who’d laughed at her and asserted he only took orders from the steward. Villagers say that after this incident, Heinz shut himself away and spent multiple hours talking aloud to himself, and upon exiting, stepped before his sisters and expatiated on human dignity, intermingling various lines from Schiller which he’d learned in school.

“Thenceforth they feared their brother, whom they’d always called ‘little one’ despite his fair height, and believed he’d become a man. So they avoided lecturing him and instead conducted their plans for him on the quiet.”

Here the amtmann had to pause, for his words were suddenly buried by strange rattling throat sounds, and everyone amusedly regarded the innkeeper, who folded his hands and snored the sleep god’s praises from his heart’s depths. But it seemed the sudden silence disturbed his inward devotion. He straightened up in bewilderment and asked, yawning, “Refills?”

And while the forester hastened to empty his untouched glass, the bookseller took the opportunity to get a word in. “You’re really quite thorough today,” he crowed over to his friend, waving his cigar’s smoke away with cantankerous gestures. “Really—where’s this all coming from? And more to the point—before—what you said about cultivation—”

But the amtmann brusquely intervened, “Really—whoever finds my narrative approach disagreeable, you needn’t listen. And more to the point—I can just as well stop!” And he withdrew angrily into the giant corner of his sofa.

The innkeeper, just returning with the forester’s freshened glass, thought the anger was against him. Halting in the middle of the room and awkwardly rubbing his backside, he suspired apologetically, “Come, don’t take offense, amtmann! I heard everything—things get familiar—it’s best to nod off now and then.” After this exertion, he sighed and sank back into his armchair.

Now the bookseller too went to console him, which, alongside the amtmann’s penchant for storytelling, drew him back from his corner. And with a quick sip of his beer, he prepared to continue unravelling his story’s inner threads in his roundabout way, whereupon the forester nodded contentedly. “On the quiet, right?” he prompted the storyteller in a grumble.

“Right! On the quiet!” confirmed the amtmann. “And it happened thus that they brought Heinz to my father. For gradually it might have dawned on them that squireship entails more than snobbery, imperiousness, and thriftless pleasure-seeking. The steward’s growing corpulence and his wife’s silken fineries might have made this enlightenment obvious for them. Besides—or moreover—their estate was strong enough to bear several paunches and silken dresses.

“So one evening, they’d ridden in ceremoniously and, with many blandishments and courtesies, explained their situation to my father, whose agricultural proficiency was widely known. Finally they begged him, nearly crying, to relieve them of their plight and, for God’s sake, move Heinz to hone his economical skills on our farm—as they expressed themselves.

“While their character and everything that was said about them didn’t particularly thrill my father, he pitied the beautiful estate. Moreover, since he’d heard from me that Heinz was a willing fellow provided one knew how to deal with him, he assented tentatively and had me fetch him.

“Which I did gladly, especially because Heinz had always shown me a certain trust. And when he’d gotten into one of his eloquent states, none could get through to him better than I—both at the school desk and still later when I visited him on vacation. Back then my supposed ability to steer men to my will was a matter of sizable pride for me, though truthfully that comprised little more than them answering their own wills.

“Such was the present case. For Heinz was essentially well pleased to get to concern himself with something new, and he barely needed persuasion. I, the witling, delighted that my mission had succeeded so splendidly. And thus, he came to us.”

The amtmann swallowed a sigh. “Yea, so it goes!” he said, brooding. And the others nodded unanimously.

“All right, so”—he recollected himself—“so, as I said, now he was with us. And he was somewhat shaping up. At least he did anything assigned him, though scarcely with real progress. And when he thus perambulated farm and field semi-actively with his long blond hair and black furred cap he almost never doffed, he bore a greater resemblance to a ministerial candidate seeking angels in the clouds than a farmer’s son.

“Still he managed to win everyone’s ready tolerance; young women especially cast him plenty of stolen glances, all the more willingly when, cleverly, he seemed not to notice. That is—I shouldn’t say—not that I suspect him—no, he really intended nothing like that, and there was nothing of pride or guile in his reticence.

“But maybe—what I wanted to say—it was precisely his aimless nature which, alas, gradually so earned her faith as in none else. Yes, it was really a curious thing how both would commonly walk around each other daylong as if each inhabited the world alone, and then suddenly come together and hold a long conversation as if they’d discovered eternal bliss in the interim.

“Yes, curious”—he repeated lowly as though he were still contemplating why that hadn’t worked for himself.

“Well! I’d say their natures matched excellently,” remarked the bookseller with utmost importance, rubbing his knees impatiently; evidently he was burning for his friend to let out the love story. 

The storyteller, however, had other intentions. “Do you think so?” he uttered dryly. “Sure, they were similar. But I think that as a man, he—in short, his handsome, casual mien’s eternal composure irritated me.

“Yes; handsome—that he was,” he conceded to himself with partial disdain. “That is, one should better, more accurately say: so little strength lay in his grace. And under the smooth surface lurked all manner of convoluted dross. Real, honest wrath, for instance, was far beyond his soul’s capabilities; at once it degenerated into a rage that appalled everyone present when one day it overcame him in our company, too: the little piece of fur atop his cap was secretly cut off and he blindly assumed that someone had wanted to offer him an affront.

“Though I knew considerably better.” The amtmann let out a forcible laugh; it almost sounded like a groan. “Oh, yes!” he affirmed bitterly. “The divine gift of sharp eyes is an accursedly useful one. That same morning, by partial coincidence, I’d seen Marie behind the garden fence pulling the little thing out of her pocket, regarding it with a gaze—with a gaze and a smile—as though it were a rosebud close to bursting. And since I knew her well enough and, for better or worse, had to recognize that the field yielded no wheat for me, I quietly gritted my teeth—and hung up my hopes—and toward the evening confronted Heinz—and told him what I’d seen and that he owed it to the girl and his own conscience to take immediate leave and return home, be he loath to.”

The amtmann’s hoarse voice sounded even more monotonous than usual while he put together this cumbersome sentence. Now his fingers trembled lightly as he wiped with both hands the sweat from his eye sockets. The forester looked rigidly into his glass. The bookseller examined the room’s ceiling with disappointment, though this time he too was silent. But the innkeeper, likely still believing he had something to atone for, wanted to offer the orator a word of consolation. He smacked his fleshy hands on the tabletop and exclaimed approvingly, "Good, amtmann! Very good! Householders’ rights!"

Here the amtmann couldn’t help laughing along, and the innkeeper admired himself duly for his excellent remark. And after everyone had loudly and clearly clinked and quaffed their hearty pints and, whistles wet, reconvened with several harrumphs, the storyteller proceeded with a stronger tone and the same repose.

“So,” he began, “that’s how the two became a pair. For Heinz didn’t show my eager disclosure the slightest surprise. He just nodded to himself as though he’d long expected that, gave me his hand, and said, ‘Of course!’ That was all he said, even though mere hours ago he’d raged with such intensity, there was froth on his lips. And Marie—no, I should say: even our people seemed to see nothing surprising in the betrothal, although it was highly unusual that a respectable, wealthy heir should wed some poor, vagrant maid.

“Only my father shook his head initially. But when her guardian, the priest, hastily sent his blessings in a letter, he too let events take their course, especially as the two hardly comported themselves differently than before their marriage. Marie carried on her work and Heinz still puttered dreamily about farm and field. Although occasionally it seemed to me that he was brooding over some decision, otherwise it almost appeared as though it should stay this way forever.

“For when, at times, they joined hands so trustingly and conversed at length in some hidden corner, one could rather suppose them to be young confirmands talking about confession and communion and such secrets than two mature lovers discussing their future together, so that I became less embittered to feebly witness it all than I’d thought in the beginning.

“Moreover”—the amtmann paused awhile and then raised his head as though by order, and his voice sounded stern and harsh—"moreover, in my childish arrogance, I sought proudly to console myself that I’d once again succeeded in giving fate a hand. I had no doubt that in time Marie would make a useful fellow out of Heinz. And presuming to have occasioned two people’s happiness, I concealed from myself the wretched incapacity of my one true desire and took my self-importance for selflessness.

“But after several weeks, amid all my blindness, moments of disquiet came oftener and oftener, especially when envy drove me to stalk the pair. For as the scripture says, the heart is both deceitful above all things and desperately wicked. And silently I was already deliberating what unsuspicious lie might determine my father to  let me travel abroad again for longer.

“One day, my heart would be unburdened when I overheard Heinz indicate to her that he didn’t want to stay with us any longer, whereby he urged her most imploringly to at once retire from her duties and move with him onto his estate. It must have been entirely without notice that this whirled into his muddled mind, or at least it sounded through the barn door as though it were his first time proclaiming it, for Marie was immeasurably astonished and raised some objections.

“Almost even greater was my own astonishment at the artful persuasion with which this distrait hero suddenly emerged—at what lovesome words he knew to express how well his sisters treated him and how his decision would delight them. And as he still painted the surprise they’d show his so unexpected arrival as a happy bridegroom, depicting it in such tangible detail as though he’d already experienced it, and all the while locked eyes with Marie like a drunkard, she listened with unaverted eyes and stood and smiled, spellbound and apparently incurious about why he’d never said a word about her to his own sisters-german.

“So, come morning, they really did declare their intention to my father. That is to say”—the orator corrected himself hesitantly—“I did; Heinz had taken me aside beforehand, confided in me what I already knew, and asked me to accompany them as their spokesman because, as he believed, they weren’t skillful enough in such matters. And though I secretly believed I was gradually obtaining a clearer insight into his crazed nature, and close though I was to regarding him a wily ne’er-do-well, I wanted to spare the girl’s peculiarity as much as possible, for I was well acquainted with her clumsiness when it came to discussing herself.

“And since, still,”—his thin voice sharpened—“my pusillanimous conceit spurred me to keep playing the magnanimous part before others as before myself, it came again that I appointed myself, or rather let myself be employed as the champion of others’ fates—this time though with the uncomfortable understanding that compliance doesn’t equate to benevolence.

“Nonetheless I must have argued the issue emphatically enough. And my father, likely not wanting to obstruct Marie anyway, might have even been basically glad to be relieved of his partial responsibility for the strange couple. He complied with her wishes more eagerly than I’d expected from him and even gave her leave to quit the estate before the end of her service period.

“Heinz, however, seemed to have taken that for granted; as I noticed after the negotiations, he’d already packed his dunnage overnight, and that very morning Marie too readied her trunk for travel.

“About this, we men of order found ourselves a bit out of sorts, particularly as it was Saturday, we were in the middle of harvest, and so could use twice or thrice as many bustling hands. But Marie suddenly seemed influenced only by her new plan, and it was as though her whole activity were done under a spell. Yet since my father had accustomed himself to leaving her alone in all things—and didn’t wish to wrest and break his own word either—he kept silent about her alacrity. But in the silence, her preparations surprised him just as they did me, for the draft animals were all engaged afield from dawn to dusk and we couldn’t understand how the two meant to transport their belongings.

“But come noon Heinz explained that they purposed amid the fair weather to traverse the one and a half miles afoot, and that the next day he would come collect her things in his own wagon. He must have puzzled that out for his surprise plan beforehand, for overall he was quite at ease.

“Then in the afternoon Marie still nobly contributed her work, so that my father, after some grumbling, swallowed his mild vexation and at my request had the workday’s end tolled somewhat early to honor her. Then she donned her Sunday dress and took leave of all the servantry, which assembled below in the farm, and lastly of my father; as for me, I’d already bidden them farewell after lunch. Yes, and Heinz—with his long hair that he was always stroking back—followed behind her like a specter and nodded musingly each time she proceeded to the next person and extended them her hand individually. But neither spoke a word, and the silence was so solemn, a few maids began bawling. And then, at last, they departed—hand in hand, like children.”

The amtmann stared into space, absorbed in the past. “Like a dream, the sight. Heinz had whittled himself a walking stick from a long branch of a hazel shrub. And as I, looking after the pair from my gable window, saw the two saunter through the gateway—him with his long dark staff, her with her colorful little bundle over her arm—I recalled suddenly the folk tune about the shepherd and the enchanted princess, and then the old song’s ending, like a dark prophecy that stiffened my heart. And thus they walked off—into the burning evening sky—shadowy black like a dynamic theatrical interlude—until the forest devoured them.”

The amtmann shook himself from his reverie. A thick bead of sweat had slowly rolled down his cheeks’ furrows. Hastily he surveyed the others’ expressions, and his gaze trailed uncertainly through the room as if he were indignantly remembering their physical presence. But they shied from looking at him; a restrained empathy welled among them, a vague grim anticipation, and the innkeeper began with folded hands to twiddle his thumbs, trying despite himself to stifle an emotion.

“Right, so”—the speaker nerved himself and quickly dried his face—“so the next day, he was meant to come collect her belongings. But he didn’t come. Likewise on the third day. And if that chimed quite appropriately with his own laxity, it didn’t with Marie’s attentiveness, so I began imagining all manner of malice, until finally he rode up on the second Sunday since their departure. And what I read in his face only made me warier; his bearing was so sulky and erratic, and his eyes looked exhausted.

“I couldn’t bear it any longer, particularly as he almost fearfully avoided me, and just as he was about to depart, I jumped into his wagon and said I would accompany him a little way. He couldn’t refuse that. But I noticed he pursed his lips as he always did when he wished stubbornly to keep his silence. And so we sat modestly apart on the narrow board without exchanging a single word, and already I despaired of getting a word out of him as I too grew stouthearted before his own stubborn reticence. Moreover the air that day was so muggy one could barely move, except now and anon we looked sidelong at each other with apprehension.

“At once he burst into a convulsive sob, and I—alas—I was but an incompetent infant, who fancied otherwise—and as I said, the air was so muggy that day—and suddenly we both melted into tears and howled awhile, each more piteously than the other. Then he told me everything.

“Blazes!” the amtmann interrupted himself and chuckled shyly while his eyes sparkled. None else moved.

“Confound it, it was a matter of course,” he swore on with increasing vexation. “The sisters and the surprise, I mean; that never could’ve worked out. And what with Marie and the new position—in short, it all happened differently than Heinz had envisaged. And as Heinz then gradually related to me the whole course of events, truthfully I could hardly comprehend how I hadn’t predicted it and how I’d let his fancies possess me at all.

“As it happened, that Saturday the pair hadn’t come home at all but on the way had the idea to stop and pick the sisters a bouquet of flowers, whereby they’d wandered deeper and deeper into the woods before finally losing the road entirely. And since they didn’t want to walk in the dark and get even more lost, they’d made themselves a bed of moss and leaves and slept soundly there till the morning. And Heinz, he delighted like a child upon remembering this silly incident.

“Then, he recounted, they’d continued their journey lightsomely and soon found the road. Thus they came cheerfully before the village, where he detoured to show her his whole estate. But Marie grew quiet, as though the riches oppressed her. Finally she took his left hand most shyly and they entered the house just as the sisters were leaving for church. And I can imagine only all too well how the poor, simple thing might have stood there in her plain peasant’s skirt, wilted bouquet in hand and bundle over her arm, before the polished gentlewomen.

“Now tears began welling anew in Heinz’s eyes as he then slowly sputtered out the rest, and in his voice stirred something like a slyly burgeoning hatred.

“But I had to assume that the sisters had already long since received tidings of their due surprise. For when Marie, in her innocence, approached them without ceremony and timidly proffered them the bouquet, they looked past her rigidly, turned away apparently unnoticing, and showered their brother effusively with caresses. And still before he could interpose a word of explanation or salutation, and since they likely saw the fury that empurpled his face, they now assailed Marie with their garrulity and addressed her like a gracious young lady, speaking of great fortune and undeserved honor but all with such contortion, only they themselves could know whom they referred to and whether they meant it in earnest or jest. All the while they brandished their parasols and hymnbooks as though to parry contact by all means. And suddenly they bowed deeply before the utterly bewildered creature, breathed an apology for sadly being unable to be bothered any longer in the lord’s service, and embraced and kissed Heinz once more before whooshing away their upraised heads.

“And so they strutted on pompously without Heinz being able to manage them. For this fear that his sisters’ behavior might veer into open hostility, which fear had so long hindered him on his estate, in nowise was coming to pass. Rather they outdid one another in the fine courtesies of their domestic commerce with the girl, and Heinz could hardly explain why he remembered their words as so derisive and hateful when in sound they seemed to exude only amity.

“And I’m still convinced myself they saw nothing base in their actions, and thus I do well believe they acted in full candor before their brother, deliberate and calculated though all the overbearing haughtiness with which they harrowed the girl certainly was. Yes, their object was only ever Heinz’s happiness, and so to avert great calamity and abject ignominy from their house.

“I also don’t wish in the slightest after so many years to conceal from myself that these sophisticated ladies couldn’t have possibly had any proper feeling for this simple child’s worth; as such they passed judgment like that huckster who appraised pearls by their weight in pounds. And by continually emphasizing that Marie was truly no maid who had to eat at the servants’ table, maybe they fancied they were conferring her a favor, while with such malicious flattery they found the most effective means to work their wicked will.

“For initially, Marie, in her heart’s unclear simplicity, had to presume in such an expense of stilted phrases a higher level of civility. And patient and hardworking as she was, she supposed that restless servility could countervail her lowliness and earn her the recognition of her distinguished kinswomen. And so by the second day, she’d already set as busily about her work as ever and attended and assisted and saw that all was orderly, which was quite necessary in that neglected household, and the people there too deferred to her every signal, though the steward sought to lodge several complaints about it with the sisters.

“But all this only served to complicate the situation even further, for now the sisters took to praising and laying excessive stress on her ministrative ability wherever possible. And deliberately lamenting with patronizing miens, especially in their brother’s presence, that she’d had little else to bring along to her future home, they thrust the girl down into the position of maid, as if unintentionally and even with the appearance of rectitude.

“Marie however grew quieter and quieter under the influence of these perplexities. And when Heinz wished to impel her to candor, she only evaded and opposed him with her gentle conviction. For her healthy self-esteem might have taught her quickly that the sisters’ airs and graces stemmed from an irreconcilable antagonism, and that whatever she did only brought more grist to their mill. And to secretly complain about this to Heinz and thereby sow discord between the siblings—that ran against her nature and maybe didn’t even enter her mind.

“And so the two would soon walk around each other, distraught and wordless, similarly to when they were with us, when they believed they weren’t completely alone or safe from spectators. And Marie descended ever deeper into her helpless diffidence and devoted herself ever more keenly, almost as if consolatorily, to her work, the more the sisters tormented her and preened themselves on their shallow cultivation and eloquence.” The amtmann’s face grew gaunter and his outstretched forefinger skewered the words, as it were.

“But Heinz—,” he wanted to continue when the bookseller suddenly leapt up as if pierced by his chair. “Listen, just let me—don’t take offense!” he spluttered and choked over his words like something bony long caught in his throat. “But really, it’s as though you wish to speak against cultivation! And the sisters—well, forgive me but you know you’ve your own bias in all this, your own rendition of events!” He looked at the storyteller with an honest indignation, albeit a bit apologetically.

A faint smile wandered over the amtmann’s weathered features, half-derisive and half-grim, before he said dryly, “Cultivation takes its own course. As for me, I’m only telling you what happened; I should hope the sort of cultivation you have in mind wouldn’t suffer from that! And by the way, would you finally shut your trap?” he added more amiably, not observing the second reproach. Therewith his critic relaxed, shrugging his shoulders.

“So, what I wanted to say: Heinz imagined despite everything—and he seemed to have considered this carefully—that his sisters had honest, benevolent intentions toward Marie, doubtless as toward himself, and just didn’t strike the right note with her. And under this assumption, he had, on the day before visiting us, awaited an opportunity to converse longer with his sisters and so clarify to them his bride’s peculiarities.

“But even an angel from heaven would’ve failed to convert them, so deeply had the conviction of their infallibility entrenched itself in these three gems of cultivation. And if ever they granted their interlocutor a word, it was only to catch their breath and then chitter their chatter at even greater length, like dogs baying on at their own echo.

“And so they’d only responded to Heinz with the same commiserative extolments at Marie’s expense, until finally he’d put his foot down and with threatening looks forbade them to bother the girl with their remarks anymore.

“At once that instilled in me a vague unease, and it was so disagreeable that I made some objections against his sisters. But he ignored that, before finally, with a notably wary sideglance, answering me curtly that I knew well or could well know what benevolent intentions all three had with him. And when I insinuated that it might have nonetheless been better if Marie were to retreat to our estate until the wedding, he refused even more monosyllabically, whereby something of piercing resentment inside him flashed through his eyes and suggested to me that by then he’d discerned in me more than I’d yet admitted to myself. Consequently I dared not press him with more counsel and accepted with relief when suddenly he bade me a terse farewell.

“And then”—the amtmann nodded gloomily to himself—“yea, then the inevitable befell her, as befalls a young plant set in hostile earth.

“Back then I was scarcely shrewd enough to have foreseen the disaster far in advance and even intentionally neglected my misgivings by hoping self-deceptively that in the end, the language of nature would have to silence the sisters’ fatuous contradiction, as dogs stop yapping upon realizing their folly. For the fact that we men are in greater love with our stupidity than is the good brute—that’s something I only learned gradually to appreciate, all my cultivation be damned”—he concluded his digression bitingly.

“But covertly I sounded out the true state of affairs from the villagers who visited us. For naturally the peculiar romance had gotten whispered and murmured around rapidly, though still with no one who would’ve spread anything unseemly about the couple. Initially all I could gather from the rumors was that Marie still persisted patiently with her restless industry, while Heinz grew more listless and dour for sheer idleness and often locked himself in his room for days or ruminated around alone in the woods.

“That is, Heinz’s order that they leave the girl alone had by all appearances entirely agreed with the sisters. And after everything they’d already asserted with their careful attacks, maybe they could desire nothing more agreeable. For the longer the poor thing had to elude her harassers, the more adept she became at doing so, and doubtless they recognized that they couldn’t persist far in their obvious pursuit before Heinz. Moreover they likely felt some fearful embarrassment before the girl’s timid fortitude, which of course had to cause them all the more discomfort when in their poverty of heart they could give no account of it.

“For always and allwhere, only the imperfections of their fellowmen had any effect on these valueless souls. And in assessing these weaknesses by the excess of their own baseness, they educed with experts’ eyes the surest guideline for an antidote, persuading themselves all the while that they acted in defense against the haughty intentions of a stubborn, beggarly proud young woman.

“And so they resorted to affecting a plaintive despondency in Marie and Heinz’s presence; with strained sighs, they enacted the tolerated and neglected ones and on every suitable occasion bemoaned that if only it didn’t contradict the good bride’s secret wishes. But from one month to the next, they impeded the wedding with their off-puttingly persistent interrogations about it and insisted upon the honor of collectively and sororally providing the future lady of the manor with her dowry.

“I only discovered that all later. But it was solely thereby that Marie bulwarked herself with her quiet bustle and, so far as it was possible, kept with the servants, who overlooked her peculiarities with great, amiable sympathy and couldn’t help but revere her ability. And Heinz, in his restless laxity, racked his brain vainly over her changing character or salvation from this misery.

“And once the sisters had thus gradually alienated them from each other and made the girl increasingly quiet and shy and made their brother so careworn that he occasionally expressed the idea that, in many respects, they weren’t entirely in the wrong, they seized the occasion to work upon him with cruder tactics.

“Namely, a rumor spread throughout the region that the siblings wanted to sell the estate. Whether the sisters had secretly circulated this with the steward’s help or coaxed Heinz by cleverly insinuating the disadvantages of joint management—in any case, he would’ve greeted the idea readily like a sign from heaven. And then he would’ve burrowed into the plan blindly like a mole without really looking around to see whether he was following his own way for his own good or just letting others lure him into a trap.

“For as hopeful as I was back then for their best intentions, today I’m certain that the sisters never once meant well but were exclusively concerned with getting Heinz out of the way and finally getting free rein for their imperious schemes. But enough. It was early December that same year when I learned from hearsay that he and the steward had traveled into the capital to conduct the needful negotiations for the sale.

“There was tell, though, of a harrowing incident that occurred on his departure.” The amtmann paused uneasily and pressed his fingertips against his temples as though to force equanimity. “Yes, harrowing!” he said, shrunk into himself as though he’d himself witnessed or dealt with it for years like an actual experience.

“Right, so—Heinz had already taken leave of everyone and was about to step out the front door when Marie, who’d theretofore stayed completely silent, at once came undone, rushed after him over the threshold, clung to his feet, and sobbed and wailed and just cried his name—over and over, just his name. And when he took her by the hands, pulled her up, and tried to console her, she begged him broken-voiced not to leave her in misery, and if he wearied of her, to just say the word—one \textit{lütt, lütt Woort}—and in a trice she’d go herself; she was just ‘a poor girl, a poor girl,’ she stammered to herself, and she glided back out from his arms and to her knees.

“But Heinz laid his hand upon her crown as if entranced and swore blind that she was his most beloved on earth and so stood under his protection and that she should never doubt him. And with glowing eyes, he looked around and threatened to strangle anyone who harmed a hair on her head. Then he uplifted her again, kissed her solemnly, and assured her upon his honor that he was only leaving for her sake and would return in a few days; afterward the villagers wondered why the two hadn’t at all conferred over the whole plan together. But all those present said explicitly that Heinz’s words caused them some trepidation, as in church, and even his sisters avoided his glance.

“His accursed sisters!” the amtmann gnashed his teeth and one could hear them grind.

“And Marie”—he continued tediously—“calm and composed as before, as if shielded by his words, gently released herself, bade a quiet farewell, and with trembling lips nodded to him once more. And then they never saw each other again.”

His listeners regarded the curious storyteller quizzically and with bewilderment; he in turn seemed to have forgotten them. “Oh, that’s right! Of course you couldn’t know why.” He took thought forcibly and, with a deep breath, raised back his slumped shoulders.

“Yes; not just days but several weeks passed without Heinz having advanced with his intentions. For back then, before we’d won the war, selling away a great estate at a reasonable price wasn’t yet as simple a matter as in our business-obsessed, above all profligate time. And furthermore Heinz, a business naif, was naturally at the absolute mercy of the steward, who didn’t exactly harbor the friendliest sentiments for the girl either. And though I can’t suppose that he and the sisters were in conscious and deliberate collusion, I do know for certain, based upon what he himself later apprised me of, that they’d issued him specific directives for the deal’s consummation, which directives at the very least made a speedy settlement utterly impossible. And since his own commercial standing was at stake too, he’ll have handled the difficult assignment all the more dilatorily and exploited his lord’s indefatigable zeal subtly for a mere cheap pleasure trip.

“But the nearer their wishes’ fulfillment approached, the deeper the sisters naturally sank their teeth in their dogged antagonism. And though they’d until then only dared torment the girl secretly, thenceforth their posturing veered into open mockery, so that the defenseless creature at last latched entirely onto the other maidservants.

“And now in frequent letters to their brother, they lamented the reservedness and diffident temperament of Marie, who meanwhile avoided more and more speaking about the sisters to him and, blindly trusting his parting words, rather stopped writing him completely, especially since she wasn’t particularly skilled at all with a pen. Indeed, the steward later showed me her last letter to her beloved, wherein she begged him in clumsily worded Low German to have patience for her ignobility and not to beset her so strongly with questions. Of course the sisters only grew keener to distress him with their letters. And so Christmas drew near without Heinz having achieved anything in the capital, and villagers widely doubted he’d return home before the second week of the new year.

“Back then, there persisted everywhere in our region the religious tradition of the lords celebrating Christmas Eve with the servants, and even the sisters, despite their enlightenment, dared not break this venerable custom.” The amtmann’s voice trembled for bitterness. “Oh, it was brutish,” he flared up, “how they abused the festival of love.

“So”—he recollected himself—“all the servantry, and the steward’s wife, had assembled for the gifting ceremony, the Christmas tree’s lights were already burning, and the sisters had assigned everyone their seat at the Christmas table and assembled their plates with presents; only Marie still stood aside and stared unnoticed at the floor, so that the room suddenly fell silent and the people awkwardly shifted their regards between the girl and their mistresses, who seemed, however, to want and be merely waiting for her. For while everyone could see how the distressed thing fought back tears and while she alternately turned red and pale, the youngest sister strode directly to her, grabbed her wrist while laughing, and pulled her to a footstool at the other end of the room, upon which was a box Heinz had sent his bride from the city. But over the box, the three had draped one of their already worn silken dresses, as at a junk market, and with expressions of condescending grace, they offered it to the poor creature, who thought she might perish from shame and disgrace, as both her Christmas gift and a piece of her dowry; they would have it renovated.

“And the agony of these moments and the melancholy of the sacred hour and all the quietly suffered sorrow of the past wretched weeks must have overwhelmed this martyred heart all at one stroke, and in her humble despair, she threw herself down before her tormentors and implored them mercy and promised them each obedience if they would please just show mercy—she was just a poor girl and would renounce their love. And the sisters stood over her, upright, and nodded and listened like judges hearing a confession and lectured her about what calamity she’d brought about and commended her, the shameless one, for finally seeing reason. And while the servants, muttering, crept out the room, away from this display of disgraceful conceit, she still lay at the ladies’ feet and cried that she was just a poor girl, a poor, simple girl, as though those were all the words she knew to comprehend all her misery and impotence.”

The amtmann’s speech grew clumsier, and one could see the effort it cost him to keep calm. “Just an \textit{arm, slicht Wiew}!” he repeated to himself tonelessly in her dialect, keeping the words for himself, so to speak. “But these three wenches,”—he burst out furiously—“they knew all manner of ladies and gentlemen—and perhaps menials and maids too; but what did they know about man and wife?” The storyteller was gripped with disgust, and he turned and spit on the floor.

“Right—so”—he tried to soothe himself—“the story is almost over. Following that holy eve, they seemed not to hinder Marie anymore, who in subsequent days trudged along helplessly from job to job, pinning her last hope—in my impression—on her bridegroom’s return.

“But meanwhile on that next Christmas morning, the sisters had written Heinz a letter and played fast and loose with the truth, as was their wont—that is, no, I don’t wish to wrong them—maybe it’s really what they thought, or what they convinced themselves—briefly: in the letter, they invoked the testimony of the steward’s wife and all the assembled servants and claimed that Marie . . . no longer loved him. And simultaneously they knew to flatter him by saying that if he hadn’t so intently, in his blind loyalty and nobility, hidden this fact from himself, he, like them, would have long recognized it.

“And Heinz—,” the amtmann struggled for words, “Heinz, whether he’d concocted this in his own fantastic brain to test her affection or those three wily vipers had also hissed it into his ear—Heinz, the hapless scatterbrain, in his sheer folly, sat down and in grandiloquent words wrote her that he was absolving her of her vows if she believed or thought it possible that she’d been wrong about him.”

The amtmann suddenly jumped for overwhelming rage from his corner, so that the innkeeper gazed at him with fright and shielded his abdomen with his hands. But this time no one laughed. Only the sand on the floorboards grided, loud and hard. The speaker stepped to the oven, in the dark background of the room; maybe that was the only reason he’d stood up.

“And Marie—,” he continued, nearly whispering, “two days before New Year’s Eve, she received the letter just as she stood on the barn floor and distributed grain among the people for threshing. With trembling fingers, she broke open the letter and immediately turned as white as a sheet, suddenly grasped it close to her chest, and, quietly lamenting, propped her chin on her pitchfork. But I suppose that nonetheless she wanted no one else to notice, for therewith she erected herself fiercely upward. But as she was feeble from sorrow and excessive strain, she couldn’t hold the jolt; throwing her arms high in the air, she fell unconscious onto the straw.

“Then the maids took her to her little chamber and into bed, called the sisters, and gave them the letter, hoping to reach their conscience. And indeed upon the initial shock, they acted visibly plaintive. When, however, the girl soon came to, they took courage and, while leaving, remarked that everything would turn out for the best. Namely, Marie indicated to them that she was feeling entirely well again, and in fact her cheeks are supposed to have glowed brighter and redder than they had in the whole recent past. And after taking back the letter, she smiled imploringly and asked that she be left alone, and then fell into a deep sleep.”

The storyteller stopped for a moment. “And while she slept”—his voice sounded stern, as though with every word he were uttering an irrevocable indictment—“one of the three had sneaked insidiously into her chamber, removed the bridegroom’s picture, which hung on the wall under a immortelle wreath at her head, and hung the letter in its place and a hay wreath thereover. Such was how the people found it the following morning.

“But in the night, Marie—well, she’ll have discovered it and been brought out of her senses by the dreadful mockery—she ran out in her shirt and likely wished to jump into the stream; indeed an old woman from the village saw her wandering about the bank in her white robe and thought her a ghost. But evidently she ended up taking fright, both at the sin and the icy water, and in the end turned back around. And the next morning, they found her lying at the manor gate, dead—apparently having lost consciousness anew and numbed to death in the winter night. Surely the people will have said afterward that she’d died of a broken heart, but of course doctors say such a death is impossible.

“But clasped tightly in her right hand, the corpse held something unfamiliar; and upon forcefully bending apart her rigid fingers, a small black fur patch fell in the snow.”

In his corner, the amtmann’s voice fell silent; the small figure stood motionless awhile at the dark oven, and no one seemed to breathe. Then the sand grided again on floorboards, under slow, tedious steps, and he sank back into the sofa. The sweat on his forehead lay so thick, like a veil of freshly fallen dew, and his sinewy hands’ ropy veins were heavier than ever.

“Damn it!” he said hoarsely. “And I thought I’d long gotten over it all.” He shook his gray head musingly while drying his forehead.

“I thought it’d been long enough since we buried her,” he added with a noticeably greater gentleness and a modest humility. “Yes, I’d ridden over too to pay her my last respects. It was an impressive interment that the sisters had arranged for her; nearly or quite the entire village was on the churchyard. And among the women, there couldn’t have been a single dry eye when Heinz stumbled as though shattered toward the coffin. And when he came vacantly up to the grave to pour the first earth and buckled up and would have plummeted into the clods had the gravedigger not pulled him back, a shudder went through the whole congregation. Only the sisters stood tearless—unmoving, like wood—and—”—the amtmann forcibly restrained himself—“yea, and yet they loved him very much!”

“Certainly I don’t wish to wrong them,” he asserted intensely. “But I saw it,” he grated his teeth, “I saw the dark delight that crept over the eldest sister’s dry face as she stooped to throw her handful of earth down onto the body. I saw it!” he raised his arm and pressed his thumb against his balled fingers as though to crush something.

“Yes!” he repeated, restraining himself. “And yet, as I said, they loved him very much.” He lowered his chin to his chest and fell silent.

Yet again it was completely still in the room; one could count the breaths, and the amtmann’s last words hovered in the space like a riddle.

That was until the forester struck his fist against the table. “Spider-pecking sparrows!” he swore emotionally, while the innkeeper clapped his hands, shrugged his shoulders, and nodded ponderously as if wanting graciously to affirm fate.

But the bookseller’s curiosity wouldn’t be gratified, and, gently grabbing his friend’s sleeve, he asked eagerly, “And Heinz?”

“Him?” the amtmann responded almost rudely. “He was an imbecile who became a raging lunatic. And since, in his insanity, he’d attempted murder against his eldest sister, they committed him to a madhouse, where he died some time later.”

From these words, the bookseller frightened somewhat, shrinking back and then keeping quiet for a few minutes; something about his friend’s story must have seemed lacking or otherwise displeasing, for soon he made another dissatisfied face and raked his hair impatiently. “And the sisters?” he finally asked meekly.

“The sisters?” the amtmann replied gruffly. “Yes—after a few years, when my father and brother died of cholera, I went up there to the March of Brandenburg to sell our estate, where I had more to do than to bother about the trio. Yet occasionally, about a quarter of a year ago, I heard them praised by a priest in the region as pious benefactresses.”

The bookseller found the way his friend told his story that day exceedingly unpleasant, though he ceased to raise any more objections. But the innkeeper seemed to remember something important, judging by the play of expressions across his face: he hung his lower lip, as when one spills a pancake over the edge of the pan.

“The men have let their beers get warm!” he groaned reproachfully and rose.
\end{document}
