\documentclass[12pt,a4paper]{article}
\usepackage[english]{babel}
\usepackage{microtype}
\usepackage[a4paper,margin=2in]{geometry}
\usepackage[T1]{fontenc}
\usepackage{fontspec}
\usepackage{ebgaramond}
\usepackage{dashrule}
\tolerance=5080
\renewcommand{\dotfill}{
  \leavevmode\cleaders\hbox to 1.10em{\hss .\hss }\hfill\kern0pt }

\title{The Birch-Rod\\ \large{\textit{A troublesome story}}}
\author{by Richard Dehmel\\translated from the German by Justin Verceles}
\date{}

\begin{document}

\maketitle

He had to laugh. If someone could see him—there, in the middle of July, screwing open the oven door. And now prodding a birch-rod into the opening! He stooped lower and delighted in how the hard twigs scraped the thin layer of ash leftover from the winter; the gentle surface’s cool ochreous color pleased him well. You lie there!

He reclosed it slowly. Yes, just what was needed: this bogey in the house. “God sees, God hears, God punishes”—he erected himself—finally he’d done away with that; now the birch might play Jehovah from the pit.

These mothers! Each one like the last. Her soul must’ve retained something of Jewishness: “You should do as your parents tell you, my child. Just you wait, dear!”

He sat back down to work. A sunbeam glared impudently from the wall and over his desk—from the image of the two. He shifted aside and allowed it to impress upon him. Hm—his daughterling looked untidy enough, there under the glass pane on that stifling copper wallpaper, her finger in her mouth, beside her gently coaxing mother. Precious, this obstinate moment.

And now his sweetheart should have these moments driven from her—the birch-rod would make a good little girl of her, a puppet. Oh, holy motherly love!

As though she hadn’t time enough to be understanding of the little one! In the whole day! While he had to toil for their bit of living. And she’d experienced it enough in herself, and also in him, that it was understanding alone—real, conscious self-reflection—which made a human more humanly. But of course, children “know nothing of themselves”—and so the mother finds it much more comfortable to beat her with a rod. As if parents should know what such a child ought and oughtn’t to do for her future.

Indeed, that was sure to inspire another tenacious battle of souls. How winsomely fine her smile had been when the other day his Polish friend, half drunk, had called him the cuckold of his consciousness. That agreed with her womanly involuntariness.

He had to laugh again. Her face—when in October she first goes to heat the house again and finds the birch staring back at her from the yellow opening, her long-lost birch-rod. Maybe it’ll be right on his birthday. How she’ll turn to him then with her golden eyes, her dark eyes, kneeling there by the oven. And her right eye, the eye of her essence, would shine large and placid for comprehension, and mutual comprehension; but in her smaller left eye, the eye of her kind, by the shadow of her feeble lid’s lashes, the womanly reproach would quiver that his deliberate silence had been meant to humiliate her. Silently around her narrow lips, a new will would dawn, down into her tender mouth’s corners; and then he would approach her and kiss her as he had when they still loved each other, before they were friends.

He rose. Just five little steps to the oven. How the narrow room had deceived him! Or had it been the Persian rug’s long middle region? He saw the colorful pattern’s wondrous twining tendrils glow in the midday sun. He felt anew his joy when for his last birthday she’d bought him the old beauty with her savings. He looked over at his little workspace and smiled.

But really it was terribly dreary, this eternal juristic wordmongery! And now amid the flourishing summer, no less.

He went to the window and saw before the gray street-front the meager poplar’s dark and lustrous greenery glistening in hot heavenly sunlight; how lonesome it looked, standing in the middle of the city. The room’s copper wallpaper stifled him more and more. Yes, he should go back out into the woods! To his father, the forester! That’s right—tomorrow was Mother’s birthday! He’d almost forgotten again.

Aye, God, his parents’ home—the oak grove; the poplar brook; the open field all around the forest’s edge; the brilliant meadows; and far on the other horizon the little farming town, with the wretched old steeple, the yellow whitewashed schoolhouse: childhood.

He sat down. His old man, who naturally would make like \textit{Rübezahl} again—as though his eldest’s unexpected visit could utterly entangle his wrathy beard. Only his steely blue eyes would suddenly shimmer somewhat darker under his silvery brows, his sharp little pupils would grow a second, his cheeks’ furrows would deepen a bit around his massive nose: “Ah, lad?”

Truthfully he still felt some guilt before this stormy red face with the dense, nearly white beard and hair, this hook-nose and stringent, searching regard, which sometimes could flash so heartily; as a child, it was how he’d always imagined the good Lord—though then still with a dark beard.

The thick folds about his nose-bridge, aye and his steep brow, he’d gotten from him; only his eyes were cut according to his mother’s, and more gray than blue, more mood than will. “Aren’t you a peculiar one, boy?” had forever been her harshest reproach; she fathomed the whole world with her forbearance. His dearest mother—tomorrow!

Oh, how her whole slender figure would tremble for warm love, for almost timid joy, up to her rippling temple hairs, her gray eyes, her fine features’ many wrinkles, all the little worry lines around her haggard mouth, the runes of motherhood. Yes, she was still beautiful, his old mother; but her withered lips were the most beautiful, so lit-up from wrinkle to wrinkle. That’d always seemed to him like the expression of her entire life—as though her reticent heart flickered deep-red in these creases, as the vestal fragrance of a narcissus’s yellowy stigmatic folds does about the fine purple seam of its pistil’s crown.

For indeed narcissi were her favorite flowers. Oh, how she could plant them! They might only stand apart, here and there, the pure, white, restful stars over the green garden lawn, so that the delicate brownish calyx was visible up on every stem, like a long Danish glove on a noblewoman’s arm. Yes, she fathomed the whole world.

And tomorrow he would kiss her, and she would fathom her peculiar boy too when he went out alone into the open air, to some corner of the woods where the shadow-swaying wind would sweep through a lupine field. He could smell it already, the sweet scent of the thousand golden-yellow tapers, lying there at the edge of the velvety green-gray sea of digitate leaves, beneath the fervid blue vault—why ever had he become a lawyer?!

A witless tic of boyhood! Merely to show his parents he could forgo his few groschen, even for the costliest study. And now, he was a lawyer. Him—with his shrugging-aside of all so-called justice. But he would become a writer yet. Devil take the clientele!

But what about his wife and child? His old man would speak anew of crazy ventures and his mother would feel sorrow again in her old age; anyway she always regarded him during his visits with the still shyness of sympathy.

So, tomorrow he would bring along the little one. She was grown enough to accompany him; then but unity and ardency would reign in the lodge, as the other day on Easter, when his wife and she had accompanied him. Then his parents would feel more like grandparents and not ask their son so many questions, so many awkward life-questions.

He stood up and opened the door. “Recha!” he shouted through the hall. Then he sat down again at his desk and picked up a document.

“Erich?” she entered, leaving her finger on the handle.

He looked up. “Where is the little one?”

“Playing—she’ll be back soon.” She held the handle tight; it sounded as though she wanted something from him.

He moved up again before the bundle of documents. How pretty it still seemed to him, this profile of her noble, Semitic nose, matched by the brown wreath of braids about her brow so regally that it made her small figure appear larger. Maybe he still loved her after all. But look out! Now she stepped behind his chair.

“Hey! Erich!”

“Hm?”

“I have to tell you something. I bought a birch-rod yesterday.”

“Oh?”

“Yes. I had no choice. Really—she’s turning too useless.”

“Detta or the rod?”

“I’m being serious, Erich.”

“I am too!” he turned around to her. “And—tomorrow I’m going to see my parents and I’m bringing the little one; please ready my knapsack for me.” She nodded. “But please, only the absolute necessities; it’s only two days.” She nodded again. “And—well, what’s the matter?” She was fighting back tears.

“Erich!” she conquered herself. Only her left eye was still fighting. He drew her close.

“Look, heart, forgive me! But really—how should I respond? You know my view! Children are not young apes—at least not once you’ve chosen to beat them as training. You call Detta stubborn, and who knows what else, because—well, because she’s in her third year. When she’s twenty, it’ll be her character.”

“But—”

“No! Enough now, please. I would’ve been something better today if my old man’s dogwhip hadn’t made me so mulish. Teach her to be dutiful as much as you like—but not with beatings, I beg you.” He pointed to his bookcase: “There! Read something about suggestion! You have your conscious will, you know.” Something like a fine smile flitted about her mouth’s corners. Aha! She was thinking about the cuckoldry of consciousness—that damned Pole! “In any case, I forbid the rod.” He almost indicated the oven.

“You don’t seem to place much value on my conscious will.”

He released her. “Blue blazes! Now you’re getting touchy!”

“Now, now,” she began appeasing—and again that flitting smile.

“And what do you keep laughing for!”

“Me?” She stared at him silently.

The door flew open. “Papa! My hans are full of sun!” the wild girl came whirling in. How the blond threads of her locks hung around her fervid dark eyes! And that remarkable, willful little nose! “Look, Mama!” she opened her fists.

“Would you like to go with Papa to see Grampa tomorrow?” asked the mother.

“No!” she stuck her nose up.

“But it would make Grampa so happy, and your dear Gramma!”

“Grandmother!” he stressed.

“No!” she stamped her leg.

“Well, then just stay here.” He took her little hands gently and slowly stroked each finger straight. “Then father will be all alone to listen to big black Juno bark, bow wow wow”—he fixed his gaze upon her—“and watch the colorful doodle-chickens play”—he let go of her hands suddenly—“cock-a-doodle-doo! And—”

“Big moo-cow! Detta will go!” she hopped and spread her little arms. “Cock-a-doo-doo, so pretty,” she caroled and hugged her mother’s knees.

She nodded to him with ready understanding. Only—again that unconscious mouth-twitching!

\null
\noindent\dotfill

\null

The cumbersome omnibus sounded really rather prehistoric as it rumbled. And the bumpy country road could’ve long used a new laying. You could get seasick on the worn-out seat springs.

He stretched himself and wanted to push his hat out of his forehead. But the hot morning sun stabbed right past the sleeping coachman and aglare into the front seat; the reddish brown of the faded plush upholstery nearly smoldered as if singed. “Sweat and dust, sweat and dust”—he heard the two nags trot their usual clap-clapping trot. The young elms at the sandy wayside looked as though they too needed shade from the heat.

“Papa”—the little girl pointed musingly at the drowsing coachman ahead—“does the whip paying with the win?” The whip wagged to and fro from the dozer’s hand in time with the horses; the reins in his other hand must’ve yielded the movement.

“No, my child; the wind has left the whip.”

“Where the win is?”

“Sleeping.”

“Seeping?”

“Yes.”

“Where he’s seeping?” Christ alive, these endless questions!

“He’s sleeping!” She really could be quite the grig.

“He’s seeping?”

“Yes!”

“Where?”

He said nothing.

How she’d tormented him on the railway with her incessant curiosity!  There, thank God—she seemed to be drowsing now too. “Black and brown and black and brown”—he heard the nags’ trit-trot again. Now she was nodding off. The whip had lulled her asleep.

He thought about yesterday. It couldn’t have been easy having her around all the time. How might his mother handle her? “You peculiar boy!”

Indeed he could put up the parasol Recha had readied for him yesterday as a birthday present; in many ways she was very prudent. He reached for the carefully wrapped gift. But the dust—the gift was for his mother! That’s not something you give at second hand. Ah, folly—childish affections! No—reverence—his mother’s birthday!

Suppose his siblings were feeling this way too? Scattered through the outlands, born of one womb, which years and decades ago today was born in another outland. Womb from womb—he looked at his child—and womb-fruit by womb-fruit. He saw nearby the young little trees dwindle past at the wayside, each eternally far from the next. He saw them close up in the distance, tighter and tighter along the alley; it led to the homeland—away from her, far, far away from her—oh, his parents’ house!

Aye, from afar now—how his heart extended to his old parents! And then, how he raised his arms high, around her neck, upon first reunion—even still. Then he was all her child, her blood, the life of her life, devoted, unwitting, as unto the heart of nature. He saw himself enter the little living room, head stooped, through the short door; he saw the linden branches tap upon the windowpanes, saw the two glossy birchen cabinets, the rifles and antlers, the pleasant green gloom.

Yet then the other life entered with him, and between them—the one with the goal-concerning questions that men pose to themselves, men in contradiction with nature, and thus also with his fellow creatures, with every immediate neighbor: the life of the transformative spirit, the finally conscious will to the future, the eternal battle for new culture.

Then he was child no more, and they were parents no more; then he was a youngster, and they the elders. Then his dear mother tongue—oh, holy word to the feeling heart—was no longer a tool for understanding. The same well-intentioned word meant something different to them than to him, try as he might in childly diffidence to heal their division. Then the cool, shadowy, stilly room could get rather stifling and oppressive.

Suppose that could happen between him and his child too? Love from afar?! Enchanting, her sleeping there cluelessly, shadowed by the sleeping coachman. And today she would surely overbridge any division. But in the future? Oh, stuff! She would walk a tightrope for him if it only suited her!

He saw the rein-ropes in the coachman’s hand swing and skip on the trotting nags’ legs. On their backs and about their sweating flanks, the sunlight danced to and fro in big bright dapples; it was unbearably hot. The three twinkling brass rings on the collars bobbed up and down with the shoulder-strap fixings—up and down, in sweat and dust. He checked his watch. Just half past eleven—about another hour. 

He listened again to the hoof-beat: black and brown—up and down—up and down, sweat and dust. Ah, now—ahead of the weary horses’ necks surfaced at least the village where they always stop. There would be something to drink. And to smoke. Oh—his cigars! He yawned and leaned back—five more minutes.

The rocking motion of the horses’ legs grew stranger and stranger; their flanks’ reflective waves swayed in near arabesques. He closed his eyes halfway. How his consciousness relished it all! The collars’ rings glittered and jerked up and down to him like three great blinding stars—up and down; black and brown; black and brown and white and dust.

He closed his eyes a little harder. The stars twinkled whiter and whiter. Up and down, white and numb.

No, that wasn’t the right word; it was yellow. Yes, yellow. Sweet, yellow, lupine perfume—so nice and cool. There must’ve been a field somewhere—a lupine field. He’d probably just missed it.

No, it wasn’t yellow. Because they were narcissi. Yes, narcissi. No, he was dreaming—no, he wasn’t! Indeed they were three great, distinct narcissus stars—dazzling white—no, five—no, seven—seven white beaming blossoms. 

Seven nodding narcissi—each with a purple and gold-colored coronet. Seven slender noblewomen, with rippling temple hairs. Oh, how beautiful! Each with such gray eyes—Mother’s eyes. Around each dainty arm a long Danish glove—yellow.

And they bowed before him, one after the other, with their radiant white hats. Each till the seventh. That one held a mirror—had dark eyes, dark brown eyes.

The first stepped forward, and she said a word to him. And it was her name, and he’d heard it before—only he couldn’t recall it. The second said her name too, and the third. And each ended with the word “meaning,” no, “being”—meaning, being—and the fourth and fifth and sixth; and the purple and golden coronets nodded. Only the seventh was silent—and pale—and just held the mirror. Which was opaque. For she shook her head, and her left eye looked sad.

No, it was far too merry—how her purple coronet wobbled. For it wasn’t a coronet at all; it was a thick red cockscomb, bobbing in the sun. It was a whole cock’s head, and thick red cock’s neck, which billowed. It beat its two glistening wings through the air and cried, loud and clear: cock-a-doodle-doo!

He opened his eyes wide. In fact—just now the omnibus struck with a heavy rumble upon the village street’s first cobblestones. And over on the one yard fence, a cock craned and crowed a second time. The old wagoner raised his stubbled chin: “Whoa, knackers!” cracking his reins on the horses’ sweat-shining thighs. Slowly the little one too livened up.

What could the dream have meant? Oh, nonsense—as though it should mean anything! But where had it come from?

Maybe—the cuckoldry of his consciousness? Hm . . .

The Pole’s remark must’ve penetrated him more deeply than he’d thought.

\null
\noindent\dotfill

\null

The evening sun seemed nearly to crook today—for thirst. The copper-red ball grew ever thicker, there on the horizon behind the marshy lake’s mists. Right between the two thickest old poplar trunks by the little road bridge beyond hung the dark-red behemoth in the hoary distance, directly below the black-green canopy’s trembling edge.

He’d never seen it set so large and lusterless. Only the three broad wedge-like refractions through which it drew water, as the people here said, streamed splendently under the purple sphere as though cut from golden topaz, showing it still gave light. The central wedge was only very short—like a mighty plinth of sunbeams. Before the swelling yellow of the sidelong slants stared both poplar posts deep-black with their barken margins. The canopy grew darker green.

“Tomorrow will be sweltering too,” the old man stepped out the open front door and approached him at the garden fence. “All my young pines will wither—awful year!” He pointed with his pipe to the acacia branches over them: “The leaves are already falling.” The tobacco smoke swirled up and touched one of the wilted racemes.

“Do you have any new beehives, Father?”

“One”—the old man took the bench by the fence. Now he smiled and pointed to the little girl, who was crouching at the front door’s high threshold beside “Lotte Goldsnut.” The dachshund lay flat and lifeless in the warm sand with all fours sprawled while the little girl tried painstakingly to stick each of three fallen acacia leaves between her crooked forepaws’ four toes. Whenever she finished one paw, the dachsmadam would swipe the leaves off again with the other paw, and the game began anew in earnest. Just what did Recha want from her! She was an incredibly good girl.

Now his mother stepped out the door, carrying in each arm a big bowl of soured milk. He leapt to her assistance. How her wrinkles all delighted, and her eyes expressed kisses when he took a bowl from her and set it on the garden table—a regular birthday expression! For her delight was likely also in how her eldest would savor the cool refreshment, with black bread crumbs and overstrewn caster sugar. How the fatty cream smelled of the ice cellar! The gentle felty film looked right wintry.

“Now, dear,” Mother smoothed Father’s snow-white hair, “should I lay supper here or under the linden?”

“Better here, Mutti,” he preempted his father; “The sun is so pretty here.” The red disk now met the landscape’s horizon; the cloven beams had disappeared.

The old man grasped his beard and clearly growled to himself again, “Sentimental dross!” That was his favorite trump card.

“The linden blossoms smell too strong anyway,” said his mother quickly; “In the evenings it can get numbing.” Now she stooped down to the little girl: “Eh, lamb?” she softly stroked her hairs from her forehead, looking fondly to the old man, and went back in the house. Lotte Goldsnut rose.

“She has a sensitive nose, your Mutti,” said his father, grabbing at his own promontory, and puffed away a thick cloud; “She gets nervy.”

“Grampa”—the girl came tottering from behind the dachshund—“are you Santa Claus?”

“Could be, my little mouse!” he nodded amusedly. She looked pensively awhile at the one bowl, through whose greenish glass wall the white contents shimmered. Then she returned to the threshold, where the faded acacia leaves lay in the dark sand.

“I’ll be back”—the old man rose—“I must check if Juno’s in the yard; the cur strangled one of my young roosters recently.” He stretched himself. “And I’ll coop up the flock.” He entered the house. Lotte Goldsnut wobbled after him.

Only the top third of the sun was visible now, like the bare red eyebrow of a great blackcock. Now it was eclipsed, nearly concealed, by the distended udder of the gray bell cow coming with the herd over the pasture. The country road’s dust ascended around their heavy bellies. The town hospital herdsman limped barefoot after them. The copper bell around the lead cow’s neck rang out through the bridge planks’ hollower sound. The herd was too full-fed to low. Their mouths were still chewing.

Now the sun was but like a glowing bow of eyelashes; that probably owed to the faraway rushes and reeds. You could almost see them slowly submerging. He threw away his quenched-out cigar and leaned further on the fence. Now the last streak died away, just above the poplars’ floor, as though shriveled in. Suddenly it brightened a little. The pallid misty veil seemed to cool down. The dull red-gray color slackened gently into green. Through the silent poplars and along the stream from head to head ventured a breeze, still with apprehension. Now—the sluggish leaves began to whisper.

He started—an overlate bee passed him from the linden and toward the hive. Suppose his father felt nature’s festivity just as particularly? With such sensuous devotion? No. That was modern thinking. Modern sensualism. Modern science, too.

Moreover the old man probably devalued these sensations through habit—sentimental dross! Though, he’d once heard him say, “The high pine forest, though snow must lie, that is my church!” But that was just it: church—an unnatural thing! There—the poplar leaves beyond, up at the highest point, how they hung darkly in the pale airy blue, every edge beset by a gentle, trembling sheen—was it not a deeply festive thing to know that now the last splitting sunbeams broke alow, bright-gold, through the warm leaves’ breathy haze in the evening coolth?

“Papa”—the little girl shuffled carefully onto the bench, holding out her apron, which she’d filled with gathered acacia leaves—“are the trees tired?” Her eyes looked wide and dreamily past the table to the poplars. “They ’tand like men.”

He had to nod, wordless. Like men! Oh, mouths of babes.

He’d have to tell the mother that one; that was a word from her blood. The girl still sat dreamily; quietly he entered the hallway. And he should tell her about the narcissus dream too! Aye, and help his old man lock in the chickens! He always appreciated that greatly.

The kitchen was open. His mother stood at the stove, turning a pancake in the skillet. No, now wasn’t a good time—better tomorrow morning. “Ah”—he couldn’t help breathing in the sizzling cake’s scent.

“My big boy!” she took him by his bearded chin, caressing every word. “I’m sure you’re hungry after your long walk.”

“Where is that dog hiding!” his father came from the yard; so helping was out. “Decides to start hunting in her old age—I’ll have to sear her with shot if the whip won’t work anymore.” He was reddened with anger—like his roosters. “You didn’t see her this afternoon?”

“No, Father.”

“I can imagine,” began his jibing; “All you do is lie in the grass.” What did that concern him!

“Ready, children”—his mother took up the setting, handing him the plates. His father followed with the pancakes.

Thank God for the little one! He sighed and stepped outside. But dear, that was a fine mess! His daughter sat upon the middle of the table and tried painstakingly to lay the sandy acacia leaves in pretty round curls upon the soured milk’s felty white film; she was about to take up the second bowl.

“Hey, you brat!” He thought a moment: no—no outbursts! What cause was there anyway? The matter was actually laughable! He approached her: “That was very impolite of you!”

She looked at him wide-eyed. “No it wun’t!”

“Look at that!” exclaimed the old man, and a hobgoblin looked out from his eyes.

Was he trying to deride him? Now wait a moment! He put down the plates and went up before his child: “Come down!”

“No!” she stiffened her arms. Oh, would that—

“Look at that!” came once again from the entrance. “Someone ought to teach her respect.”

“Someone ought not to! I don’t hit my children!” Damn it—how had he let that escape him? He never should’ve brought that brat along!

“Well,” his father growled, “the dogs like soured milk just as well. Come, Lotte.” He whistled and the dachshund came crawling through the fence. Just what had gotten into this chit’s head!

“Come now, my little lamb,” now his mother intervened. The old man stroked the dog, who was lapping up the cream. “Come, my little lamb.”

“I don’t wanna!” she properly bucked off now, clamping her fingers around the table’s edge. His patience was wearing very thin.

“Now, heart,” his mother beckoned on, “you’re not getting peculiar again, are you?”

Ah—so she’d gotten this way in the afternoon?! What should his father think of him!

“Papa dun’t hit”—she clamped harder.

Devil, now that was going too far! “Won’t you come down already?!”

“No!”

“Detta?!”

“No!”

What a grip! Just wait, beast! Is she flailing? And kicking her legs? “Leave me, Mother!” he cried furiously. And how her bare flesh contorted! How his hand burned! How the imp howled! Just wait, you little Satan!

“Well! Isn’t she a rowdy one?” he suddenly heard the old man—as though out of the mist.

“Scoundrel!” he gasped out—“Out!” and took thought. The flesh had gotten bright red—like a cockscomb. And—the cuckoldry of consciousness! The blood shot into his temples—aye, like a box on his damned ear.

Had she deserved it? something in him asked. She stood stock-still, fighting back tears. What would Recha say? He felt ashamed.

“I already had to take her to task this afternoon,” there came from the entrance again. Blazes. “Ah, forgive me! The rod hardly rapped her knuckles.”

So—“Santa Claus”—that’s where that came from?! And that’s why she’d been so oddly well-behaved?! He couldn’t help it; he had to laugh.
\end{document}