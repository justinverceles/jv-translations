\documentclass[12pt,a4paper]{article}
\usepackage[english]{babel}
\usepackage{microtype}
\usepackage[a4paper,margin=2in]{geometry}
\usepackage[T1]{fontenc}
\usepackage{fontspec}
\usepackage{ebgaramond}
\tolerance=485

\title{The Laughing Heir\\ \large{\textit{A comedy}}}
\author{by Richard Dehmel\\translated from the German by Justin Verceles}
\date{}

\begin{document}

\maketitle

Mr. Josua Heilbrodt was a man who liked to laugh. With his meaty right hand, which was appreciably delicate relative to his other body parts, he would grasp his golden spectacles and nudge them forward as though to give his big eyeballs and the rest of his giant pug-face space for his delightful convulsions.

He had even laughed when one night, his white dog Mohr, a bulldog of the purest breed, had chewed up from cover to cover his library’s most treasured possession—a collector’s edition of Maupassant in strict chronological order. To be sure, he gave the noble brute a thorough thrashing the next morning, but in the end he just grasped his golden spectacles and was moved by his dog’s exquisite taste.

Only once had he lost his mirth and disconcertedly grabbed first at his reddish pastorly beard, then at the glistening wall of his forehead, then at his derisive lower lip. This was when in a registered letter, his wealthy Aunt Christine suddenly and without explanation terminated the monthly allowance on which he subsisted and which, incidentally, I had wanted to scrounge off, and so sitting beside him I could share his dismay greatly.

This Aunt Christine was the only woman who commanded any real respect from him, partly because of her dogged albeit needless spinsterhood, and partly because, in his view, she harbored no false delusions about human existence. Except for her senseless fondness for him—in nowise did Mr. Heilbrodt fancy himself a model nephew—her only weakness was that now and again she would \textit{spritz} a little morphine into her humble body—or \textit{sprütz}, as she was wont to pronounce it, and with a tone of refreshment. But that seemed to him perfectly forgivable; for first of all, on principle he granted everyone a piece of his joie de vivre, and secondly she suffered from gout, which he had also been visited by, although—as was natural for any thirty-year-old—for different reasons than with Aunt Christine.

Now all of a sudden she wanted nothing more to do with him; he found this even more senseless than her former angelic benevolence. But she wasn’t to be reasoned with; she did nothing without cautious obstinacy, wherein she was like all Heilbrodts. Thus his dawdling days were over with.

His friends, or the few who could appreciate his open hand and fine palate, now truly expected him to one morning pour cyanide into his exquisite mocha; for he had never known work and called it the original sin of mankind. He preferred, however, to invite us all one evening for his famous shaker punch and delivered everyone a moralizing farewell address, whereby we, his siftings, had to laugh even louder than he. The next day, he sold his much-admired housewares, kept only his books, his select Wedgwood crockery, and his white dog Mohr, and became a life insurance agent.

Therewith he seemed like a new man. His impressive ugliness already perplexed people as it was, and now an irresistible eyewink helped him all the more effectively. In fact he only winked from suppressed embarrassment; for it wasn’t his nature to talk, though the words came easily to him. But it always gave the impression that this winking came from a truly fatherly heart and secretly imparted to the addressee some wise counsel. And if such a well-advised fellowman, especially a family man, was still left scratching behind his ears, he confessed with immense dignity his derailed bachelor’s fate as a cautionary tale against founding the wellbeing of one’s children on wealthy aunts, whereupon he nudged his golden spectacles and the two had to laugh outrageously. Then it was decisive, and the contract concluded successfully.

Ultimately his new lifepath, notwithstanding all the various staircases he had to climb, didn’t seem so uneven to him at all, especially since the concomitant movement made him somewhat slimmer around the hips, which facilitated laughter. Even his gout appeared as though to have blown away; and his unusual success with anxious family men determined the insurance company to hire him permanently after just a year, with the prospect of “upcoming” advancement. As he related this to me, he nearly lost his spectacles, so much did the director’s wording rattle him.

Two days later, he summoned the others back to celebrate his upcoming advancement with a little oyster breakfast. Just when we were about to fill with a janissary march of emptied Moselle and Cognac bottles the modest room that he had inhabited since his heyday, the bell rang in the corridor, and forthwith the landlady handed us a black-sealed letter. “From Aunt Christine,” said Mr. Josua briefly, and his eyes winked hard. Hereupon he dealt his dog Mohr first a slap in the face, then a caviar bread roll, slowly broke the seal and letter with an oyster fork, and to our uproarious chorus hemmed, “Aloud!”

With ever oftener eyewinks, he read as follows: “My dearest Josua! I have to my pleasure learned that you are now a man who understands work, and that you’ve gotten somewhat thinner and no longer suffer from gout. But since I don’t wish to torment you needlessly, nor myself, this morning I appointed you among my sole heirs, and in the evening I will \textit{sprütz} several grains more of morphine than usual into my suffering body. I wish to be cremated and have my ashes kept in the antique bronze vase that stands to the left on my writing desk. Your superfluous aunt, Christine Heilbrodt.”

Mr. Josua was silent; we as well. He stood and nudged his spectacles. Then two tears rolled down his face which he tried vainly to laugh away. Such a laugh he had never uttered before.
\end{document}