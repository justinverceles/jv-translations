\documentclass[12pt,a4paper]{article}
\usepackage[english]{babel}
\usepackage{microtype}
\usepackage[a4paper,margin=2in]{geometry}
\usepackage[T1]{fontenc}
\usepackage{fontspec}
\usepackage{ebgaramond}
\usepackage{dashrule}
\usepackage{verse}
\tolerance=5245

\title{A Violet Bouquet\\ \large{\textit{A letter from Hamburg}}}
\author{by Richard Dehmel\\translated from the German by Justin Verceles}
\date{}

\begin{document}

\maketitle

Dear soul! I’m sitting in the hotel. The scent of a violet bouquet pervades my warm room. One last curly little cloud ascends from my morning cigarette. The pinewood crackles in the white cocklestove—almost as though I were sitting at home. The waiter just left, almost inaudibly; only his key ring is still tinkling. Meanwhile I flirt with the inkwell and go to confession.

It’s so sweet being forgiven by women, so that one should like to sin against them . . . Too jokey, soul—be serious, spirit! Who resorts to jokes to hide severity betrays a shameful conscience. Irony is but a mask for disappointment . . . Shame is the most dangerous virtue—a mask that all too easily conceals and disguises us from even ourselves. Who wishes to discover his inner beauty must have a naked conscience . . . Only not those external ideals that stray like comets through the sky! Night crawlers have thin legs. The universe is anchored in your soul . . . Weak hearts slobber over stars. There they hang and shine, so boundless, so high; and none is responsible for their not falling in anyone’s lap. Indeed that could be deadly . . . Every power commits its excesses. Who grieves over them loves them still—like wayward children. Tire of it, then give it the knife!

Ah, that reminds me: I’ll cut myself a cigar now. But won’t that smoke the violets? Forget it then!

You’ll suffer; everything that is strong awakens the sidelong look of weakness. Make that the keeper of your conscientiousness \mbox{. . .} Wish not for fortune; it already lies within you—becoming conscious, that’s the art . . . Youthful ideals are dice games with old cups. And the old—aren’t they younger than you? And don’t they keep getting more childish! By the wisdom and error of your fathers, your youth is older than theirs was, and riper for the future—wish not for fortune; you grow it within yourself . . . And the others? Encourage them! Become ever more naked, you loveful one, so they might see your uglinesses as you do and grow truer to their own beauty and regard desire more highly than fortune. For we all love our future, yet nearly all are patients of the moment . . . But don’t boast your blemishes, like the guttersnipes with their ripped trousers, that you don’t embrace them and your truth might remain your sacrifice before the throne of clarity. 

No, that won’t do; there’s no freedom in rumination. And always with this foreign outlook—between every thought and sentence wedges another tree or oriel, and the dance of beguiled senses is commenced. Strange city!

Klopstock and Heinrich Heine. Woolmongers’ c\\ounters and chinaware storehouses. The swans, the Jungfernstieg, the famous graves. The secret loves in public houses. The harbor, the stock exchange, the Great Fire. Three-hundred-year-old gables, electric moons. The Hamburg Dramaturgy and world-renowned tingle-tangles. And so on and so forth. Oh, you great, vibrant, curious river city!

Already upon entering Klostertor Station—what a whirl of memories, dreams, wishes, and expectations the shaken soul is wheeled into! And the dream-web spins thicker and thicker, the further all your senses open up. Even on the streets yesterday noon, down through all the clattering din, amid the crying chaos of trading time, I kept hearing some distant, magical sound, like the long solemn undertone in the surging of a waterfall. And now, here before my room: the glistening basin of Inner Alster Lake, the tall electric lamps all around—each reflecting the morning sun. One never wants to leave his window—scene after scene.

Just two nights ago, when my droshky suddenly veered around the hotel corner, out the ravine-like alley and onto the broad promenade—horrific though the candelabra’s boasted cast-iron gallows-arms were, from which one could hang up all the herring barrels in the free Hanseatic city and still have room left for their corpulent owners; but nonetheless, it was this pure blue wonder-island, this far-floating festoon of hundreds and hundreds of shining globes, like a fringe of giant pearls looped through the bare lindens’ glimmering branches and around the floating isle of shimmering mist, forming a fence of flaming swords down into the benighted watery quadrate, pale as opal, entwined by swaying silver tendrils, opened darkly eastward. And ever unfathomably deeper does the twinkling grillwork bore, rest and sink and rise and plummet does the Edenic hortulan fulgor. And suddenly, in point-blank haste: two great glowing beetles shoot through the black eastern gate, one rutting-red and one greedy-green, and behind them a dully rolling hiss, a scaly nacreous snake through the water, a wailing whistle—ah, a bridge arch—there must be the Lombard Bridge—and thereon the express train that brought me from the south and now raced on northward, into the fairyland:
\settowidth{\versewidth}{“Schleswig-Holstein, embraced by the sea,}
\begin{verse}[\versewidth]
“Schleswig-Holstein, embraced by the sea, \\
“Schleswig-Holstein, one kin!”
\end{verse}

Children’s tunes in my head, I leap out the coach, bestowing upon the kind porter the honor of paying, which he’ll repay me doubly tomorrow—on the invoice. But in exchange one gets to live on silken carpets and velvet armchairs, as well as in a so-called free imperial city, which of course provides ample food for thought to an outlawed German poet:
\settowidth{\versewidth}{Splendid—aye, we scribes of souls!}
\begin{verse}[\versewidth]
Splendid—aye, we scribes of souls! \\
Splendid is this idle pine! \\
But the youth, and womenfolk— \\
At last, one is a soulful scribe.
\end{verse}

I must’ve sat awake a whole hour, relishing the synthetic moonlight, and felt increasingly derailed as a visiting Berliner. Oh, you regular barracks complex of a German imperial city! Not even electricity could cure it, this beloved Prussian parade ground of a metropolis. Indeed, we have “our lindens.” “Ethereal!” throats the assessor on the evening’s promenade. It pleases him, seeing these three columns of Siemens lamps, these round milky faces lined up neatly at equal heights, rather like a company of recruits, if His Honor the lieutenant would deign a joke. But it fits: we have style! An explosive sense of style! From the cannon memorial on the King’s Square to the Kaiser Wilhelm Bridge with its wonderful stalk-like candelabra. Not to mention all the pretty tin soldiers, mounted and unmounted, before the castle terrace and Reichstag palace—or even the grand dress parade, with its stringent historicism and genuine marble, in the world-renowned Siegesallee: the Avenue of Victory! Long live the uniform!

And with a few pious hopes, I climbed into bed. And toward morning I dreamed that the Mongols had come to Berlin; and the Spree nearly overflowed with all the glorious triumphal monuments by Prussia’s best-known graybearded masters. But on the riverbank stood all the young German artists—there weren’t many—and held their meager bellies for laughter and slapped their thin thighs for amusement, these unpatriotic “starvelings”; therefrom I awoke. But it was only the clopping of horse-hooves on the asphalt pavement before the hotel. Alas, such dreams one has in the free Hanseatic city of Hamburg; and what can you do about your dreams?

Today, certainly, it’s almost too bright to dream—especially when you’re on the verge of shoving off to the land of the dollar. The first clear autumn frost—all the mist fell overnight. Down by the basin lies a thick rimy pelt on the wooden pier. In the north, a soft green streak of cloud makes the white sky appear even cooler. And over on the Jungfernstieg, a squalid little girl is standing barefoot by the iron railing. She shivers and watches the glistening swans, the lightweight steamer dinghies with their colorful flags, how they row up and off in a circle, drawing long shiny furrows pairwise through the inky water. Poor little yearner! Wouldn’t you like to stand together this evening by the harbor:
\settowidth{\versewidth}{Ashore, aboard, America—}
\begin{verse}[\versewidth]
Ashore, aboard, America— \\
Ship ahoy, tra-la-la!
\end{verse}

Aye, and there in the Alster Pavilion is where Heine sat to escape trading time and dreamed of—frozen swans. Now the little one runs past it; the wide glass front glisters in the dawn, like a stilly pond, flickered over with ice crystals. I look east: two steamboats sound their shrill whistles.

I look north—north-northeast—what cold, hard splendor! Ahead, a colossal pedestal, the three dark-red sandstone arches of the Lombard Bridge. Therebehind ascends steeply the surface of Outer Alster Lake, a sleek silvery wall between tall red and white pennant poles; all around, into every last umbered tree-leaf, the pallid morning sun whitewashes a frame of golden lights. All is finely crowned afar by yellow-lit villa façades and the bright blackish tips of towers and oriels; in the far distance fly long thin puffs of smoke, stack after stack, like bare narrow flags fastened before the pale blue sky. There dwell Hamburg’s stiff-boned richlings.

The violets’ scent pervades my warm room; the flames flutter and hum in the cocklestove, but I’m shivery. If it were yesterday again, yesterday morning, when all this was hardly visible for the thick, rigid fog—oh, you blue, blue, gray Monday!

How strange-spirited that all was. How early the violets scented. Ah, much sweeter, much more forbidden a fruit than today. Suppose she has the letter now? Suppose she’d suspected it Sunday evening in the train station when she threw the violet bouquet after me, just at the last moment, as the train had already begun rolling, through the open window. How the old forester beside me grinned and grabbed at his heavy mustache! While I had tears in my throat. Suppose I should’ve rather ripped up the farewell letter?

No! It couldn’t continue that way. For I love the other still—and it’s a much dearer, much more soul-worthy love. And it’s a strange thing about loving twice. Suddenly you see that you only love yourself, only your little fleeting desire, and then comes the dread of whether you love at all; for who knows what it even is, this groundless, greedy self! Only occasionally do we like to send each other violet bouquets, bashful flowers of young love, and to remind each other of that summer night under the young oak tree, how we laughed about the stars, at how they peered so sillily through the dark leaves—oh, Prickle-puss, you lovely little thing! 

And so I sat and wrote fraternally to the girl who’d become too dear to me with her spirit for unimpeded passion. And it was very pretty, what I wrote about the new paradise of innocence, where there would be no more men who can hang their desire with indeterminate, indiscriminate, animalistic freedom on two deeply dissimilar women, and no more women who might find parallel pleasure in a single man. And she “shall now become my sister”—so closed the sermon—“for the sake of a nobler freedom: a freedom that sets itself its sins, sins of the heart’s lusts against the spirit of the future.” But outside, as aforesaid, lay a thick misty burden on the inky quadrate; and the Alster’s dark, leaden waves lumbered restlessly to and fro and up and down in the damp wind, like a slew of beating hearts. And the swans swam in the dense haze, small and white like crumpled paper, discarded love letters.

I wonder if she’ll believe me—if she’ll think back to that evening hour when the three of us sat together, wordless, till with stout lips and a privately doleful voice, she began her North Sea song—about the two royal children who could never see each other—while I pondered at the piano:
\settowidth{\versewidth}{For all the sin-sweet innocence}
\begin{verse}[\versewidth]
Though that should we sleep, \\
Below the overdeep \\
Expanse which sunders us \\
Would slumber recompense \\
For all the sin-sweet innocence \\
’Mong both of us \\
—Yet along with life \\
Would slumber, too, the strife \\
Among us . . .
\end{verse}
No, I didn’t feel quite well as my brotherly letter now fell fluttering into the mailbox. And the steam whistles in the basin practically screamed, almost derisively. Get thee to a nunnery, Ophelia!

Suddenly—a crowd! Trading time just began—or better yet, took to the streets! That’s right: to Ottensen—Liliencron Castle. Indeed for at least fourteen days already, ever since he’d written me that great, long letter, I’d been looking forward to finally acquainting him in the flesh, the poet-baron, the Holstein giant, with his two dachshunds, his horses, and his boundless human heart.

So I didn’t loiter any longer by the mailbox, but quickly directed my crooked soul up at the St. Peter’s Church spire, this most wondrous Gothic steeple, which was brought by earthly sons’ heavenly yearning into our vale of tears, noteworthily in only the last century, and not at all in the Gothic one. Again and again, whenever I see the simple, slender, metallically supple, venturous structure pointing with its soft green patina into the blue yonder, I can hear an angel’s exultant violin play me a Bachian cantilena. And sinning against Baedeker, I give the stock exchange a wide, shirking berth; this temple of the modern god, always with the exact same bunglingly antique skeletal pillars, as well as these droves of believers—it really sticks in my craw, although my cultural animus has been more or less breaking me of the antisemitic instincts.

Forthwith, then, into the omnibus: the \textit{fiefrädigen}—five-wheeler—where the coachman sits \textit{oben up}—on the top—so a native tries to explain to me in cumbersome High German, half looking at my elegant overseas traveler’s attire.

And so I rumble off on my five wheels, which should soon enough fall an honorable victim to the electrical zeitgeist, through the streets and alleyways, over three or four canal lock bridges, under which the wavy \textit{Fleete} creep, as sluggish as their name, from the Alster basin to the Elbe harbor. Here and there, a pretty old gabled house, left spared from both the fire fifty years ago and from real estate crookery; now and again, an old Vierlande farmer woman, less pretty with her big rigid black-lacquered bast bow in her nape under her yellow straw dish-hat, and long thin black-stockinged calves stretching out from her short blue woolen skirt; but otherwise, the same goings-on as in Alt-Berlin’s Molkenmarkt square—only somewhat fewer police.

Now down through St. Pauli toward Altona. Here wave mariners and other traveling folk from the scum of our dear lady of the sea, the foam-arisen Aphrodite. Wherever you look, one tingle-tangle beside another, plastered with the most outrageous baroque stuccowork, but overall a nice scene, the great square with all its lindens, and to the left and right the ranged music hall rooftops’ scrolling bric-a-brac, like a long stone garland connecting the two twin towns. Not far from here, an artificial hill, whereupon—to the astonishment of posterity—Bismarck shall soon emerge as “Roland.” For nowadays memorials are made for simply thinking outside the present!

Finally—the conductor calls “Ottensen.” Now around the “venerable cemetery” with the three famous linden-shaded graves before the yellowed church wall—what would Grandfather Klopstock in heaven have to say about Liliencron’s poems? And a quick vista down to the oily Elbe, where the great ocean steamers sound their trumpets, quite martial-looking, monstrous things, with those massive foamy mustaches around their gigantic mandibles; and at once I’m standing at the portal of the baronial seat of Muses.

The Baron—thus the lady of the house informed me—“is not present at the moment” but was expected to be back soon. So I planted myself in his workroom, and the young woman brought me a bottle of sherry and, shortly thereafter, since she probably smelled the foreigner in me, the national dish: eel soup. Exquisite! I groaned and fell to; lovesickness famishes. After I restored my heart, and since the house head still hadn’t come, I immersed myself in \textit{The Heathfarer}, his last book of poems, which lay on his desk, utterly unkempt, as a stand for his lamp; beside that were a few unadorned sheets of letter paper, the cheapest kind. And altogether, the room looked rather very simple, almost austere. But, as is known from world history, noble lords have their peculiarities.

So—the lamp-stand. I leafed through. But on page seventeen I stopped short—“On Aldebaran”:
\settowidth{\versewidth}{“The peacock, gilt as brimstone, rambles round,}
\begin{verse}[\versewidth]
“The butterflies, as blue as heaven, shine, \\
“The peacock, gilt as brimstone, rambles round, \\
“The glary green that sets my turf afire” 
\end{verse}

Parbleu! That was the right mood for my Spanish wine! I read on.
\settowidth{\versewidth}{“Contempt all puckered round her lips”}
\begin{verse}[\versewidth]
“And she, the murky pupil of her eye \\
“So bound with mine— \\
“Contempt all puckered round her lips”
\end{verse}
Had he experienced that too?
\settowidth{\versewidth}{“That you but love me on this ruddy star!”}
\begin{verse}[\versewidth]
“And I, a prince here on Aldebaran— \\
“Can you hear it? Would that—no—I will \\
“That you but love me on this ruddy star!”
\end{verse}
Suppose he meant his wife?
\settowidth{\versewidth}{“The men here far surpass the men below”}
\begin{verse}[\versewidth]
“The men here far surpass the men below”
\end{verse}
I felt as though I could hear his voice—
\settowidth{\versewidth}{“More love, forgiveness, and forbearance”}
\begin{verse}[\versewidth]
“More love, forgiveness, and forbearance”
\end{verse}
From afar, as in a dream—
\settowidth{\versewidth}{“No more misconception as on earth”}
\begin{verse}[\versewidth]
“No more misconception as on earth”
\end{verse}
And as near as in a dream—
\settowidth{\versewidth}{“And yet she”—}
\begin{verse}[\versewidth]
“And yet she”—
\end{verse}
I started—the door opened. He stood before me. I think I heard my name. “Richard?” he asked and gave me his hand. “Yes, Detlev!” I gave him mine, still dreaming. A few brief words, a tour through the house, and soon we sat on horseback—he on a stately golden chestnut Berber horse, and I on his youngest black Trakehner. One short trot around the cracked, ivy-ridden balcony, through the designedly wild park, past various lively cascades, and now abruptly out into the misty field, his hunting ground, the “lyrical heath,” as he called out to me with a wink and laughed.

Marvelous, how the old fellow rode. In one moment with the graceliest courbettes, light, playful, and textbook; in the next moment in a meaningful, “proud pace”; and in the next, suddenly with the jauntiest lengthy lateral movements, like a peasant boy on an unsaddled pony. His dear heath certainly looked a bit dismal today:
\settowidth{\versewidth}{“on fen and fallow wilts the day”}
\begin{verse}[\versewidth]
“—and slowly \\
“on fen and fallow wilts the day.”
\end{verse}
Only now and again did a fine, sturdy maple with its lasting foliage or a mountain-ash with its lustrous berry bunches bring some rusty or scarlet red into the misty gray scene. Glistening on these lonesome trees, held by gilded iron twine, were white marble plaques into which the colorful proprietor had had engraved in blood-red writing the names of the few living German poets. There were even fewer plaques than trees. “For those who live prospectively!” he called to me in his devoicing accent.

Now he took a hedge, \textit{en pleine chasse}. Enchanting—in the broad wine-yellow purple-seamed fluttering coat that he’d swung around his loden hunting frock. On the top, the burnous settled in a strange foldy cap—his “Saracen,” he called it. A black aigrette blowing over his ear, held together by a large, splendorous carbuncle ruby; on the front and back a clasp of bright emeralds around his head, bound by dark violet stones. Oh, my violets! I said to myself in a sigh.

Another hedge! Far away, the forest began to emerge. We headed toward “Poggfred,” Peace for Frogs, as the Baron had christened his little summer palace—wherever the dickens it was. Now we rode under the old, as yet brown-leafed holm oak canopy. Suddenly to the right, into a long, solemn, stirless fir-lined street; I nearly lost my reins for laughter, such a curious procession of monuments had the old devil commissioned there. On both sides of the sandy path, set respectfully apart from one another, were enthroned—how should I describe them?—the literary pagods. All with their feeble legs stiffly crossed, big Chinese-like bellies, and benedictory hand gestures. Many of them sat perfectly still. Some were stretching, nodding in unison, sticking their tongues out of their heavy heads. At the very end of the avenue, shrouded in mist, Father Homer too seemed to be wobbling; but I’m short-sighted. Riveted with brass nails onto each paperboard-leather pedestal was the year in which the revered graybeard had apparently first had the honor of publication for the Christmas table. The firs soughed so deliberately, I was already getting very sleepy.

“So—now I’ll show you my park of torments,” said the Baron and swung his riding whip roguishly, breaking into a short gallop. Thank God! Though the words sounded a bit grim to me. “I receive many diverse German poets,” he turned around; “They atone for their sins here.” We turned into a young underbrush; ah, that could’ve been a laugh! A clamor and a dull clopping, as from a student fencing-hall, reached us through the thicket. The horses suddenly frightened: a dreary glade opened before us, a rollered sand-strewn surface enclosed by a bristly wire fence, and in the center a large shed with a Norwegian raftered roof, which was painted this way and that in partly the French national colors, partly the Russian colors.

Standing underneath, completely bathed in sweat, was a small company of German art students, rehearsing the trickiest fencing maneuvers for their duels with the “prejudices” of their degenerate fellowmen. Slogans were just flung back and forth, all with Latinate endings, and none seemed willing to spare a hair on the other’s head. But all were setting obsessively upon a big tin dummy hanging in the middle, and each with such gravity, as though he alone travailed for mankind’s salvation.

No—it was then that I noticed it—not all! A few stood somewhat aside, equipped with long, ornate, curiously twisted garden rakes, and took the greatest pains imaginable to rake smooth the trampled sand, following all the principles of gardening with the utmost care; but over and over the ground would get trampled again. Thereby they assumed even graver expressions, and bore a cold sort of melancholy, albeit with just as dreadful an amount of sweat.

“One might rather play tennis here!” I grumbled, quite disappointed. “Don’t chide my friends!” menaced the Baron with a queer smile, and greeted each attendant with a gleeful salute, down from his golden chestnut Berber; “They’re all such well-hearted lads, and they work so awfully hard.” “Oh, come on!” my bile finally ran over, as we hurried on; “What concern should a poet have for such prideful imbeciles? Idiocy itself would kowtow to them; and may it just—it would be at ease doing so! What should a poet care about the stomach-aches of the intellectually poor, these second-rate Hamlets? When their palates are gratified—then they’re happy. Is it the object of art to cure nerves? I sneeze at the ‘heart’ of the artist; I know his pity and I know the cheers of a stirred audience. The poet exalts his sort—the aquiline souls, brazen as sun! Us, the free lords of the future! Us and—”

Smack! I lay in the sand. My stallion had taken my arm-waving amiss and lashed out of his reins. As I arced high through the air, I saw the Baron grab at the edge of his colorful cap, laughing, grasp the black aigrette, and—huh? It looked like a schoolmaster’s rod! A tight fist seized my collar. “Sirrr”—I would’ve flown into a rage and—

—and nearly upset the table, drinking glass, and the rest of the sherry too. Blazes, so I’d nodded off from the heavy breakfast wine. Liliencron stood before me—truly this time. Then the exchange of names and: “Forgive me, Baron.” “No, it’s splendid! Exquisite! Wonderful!” And forthwith he’d returned with his short legs to the threshold, ordering a new bottle.

But—it wasn’t at all the man from my dream, with the loden coat and the hat of emeralds; here stood a nimble, stately-stocky forty-year-old in his formerly new frock coat, with the scuffed, rumpled felt, the bright, immaculate glacé leather, and fine, shipshape pearl-gray trousers.

Ah, the sherry wasn’t from the baronial cellar; it went on the account of the mariner’s pub next door, as he later admitted to me sportively, and the lady of the house was just the landlady. Even his dear dachshunds, he’d had to give up since the dog tax was introduced. The cigar he then offered me, however, was a royal one, sourced staunchly from the soundest importer in Hamburg.

There was just space enough for us both—I on the worn-out sofa, he on the edge of a commode he’d drawn out. “This is how Germany accommodates its poets,” he joked apologetically; “Well! All in due time, as we said as lieutenants; time pays all dues.” Though the prospect was quite atmospheric—out over the churchyard. Indeed the good Lord sees to that, just as He sees to the dachshunds, the fine horses, and the castles—on the moon! Or on the red Aldebaran! And we smiled perceptively at the heavenly blue smoke that broke free from our cigars’ white, ashen ends:
\settowidth{\versewidth}{Blue as heaven? By His head—}
\begin{verse}[\versewidth]
Blue as heaven? By His head— \\
Within it burns infernal red!
\end{verse}

Oh, you have my thanks, Detlev, you fabulous old soul; it was a strange, especial few hours.

How he was always getting “enchanted,” over every young German poet, every merest talent, the kind, noble gentleman—and over himself, the honest fellow. And his splendid hatred for all faint-heartedness, moral eunuchry, and showiness. And his conversation—like the distant melody of the sea on hot summer days, lying in the dunes and wishing only to hear:
\settowidth{\versewidth}{How the surges sport and spume,}
\begin{verse}[\versewidth]
How the surges sport and spume, \\
And scrub their abyssal rheum.
\end{verse}

Now and then his speech gave a cruder tone, with such a remark as “among us maidens,” but always accompanied by a tactful, smoothening gesture, as it were, and a sovereign spin of the head, which befitted well his trimmed hair and saber scar at the edge of his forehead. And then there was that innocent laugh, which would flit through his bold, as yet blond—autumnal blond—cavalry captain’s mustache, down his sparse, stocky nose, from his still, reticent eyes, which seemed to swim aloft like an ever-blue Monday, only sometimes bearing a keen flash, as when the evenstar at once disentangles from the meadowland fog. And suddenly I had tell him of my violet bouquet.

He nodded and quietly indicated the wall over the sofa, where he’d nailed up a wide arch of draft paper, painted in his chicken scratch with an adage from old Lichtenberg: “So long as we describe our lives without recording every weakness, from our lust for glory to our vilest vice, we will never learn to love one another.” That was his only apparent decor, excepting perhaps his endless, eight-legged diplomat’s desk, lent him by the Breslauer Dichterschule—the Breslau School of Poets—to the deep distress and disappointment of all court bailiffs.

Aye, and another thing, so nothing’s missing in the scene: in Hamburg’s “Golden Forty,” what the red-lamplit brown-skinned Hungarian babbled out so lovelily amid our midnight snack on secret fruits: “You little devil!” For it was the only thing she could say in German, the poor child; that sufficed for her business. Right—so I don’t lie—there was something else she could say: “Another bottle of champagne!” she entreated in a whisper. But we were steadfast—that is to say sedentary; for the loveliest things I don’t just share with anyone.

And as we took each other’s leave, we laughed and whistled to ourselves—whistled in defiance against this “prejudice,” that one, and all the “prejudices” of every possible world.

In the evening, a farewell by the harbor! Till the boatswain whistles—till the gangway clatters out. Then:
\settowidth{\versewidth}{Then fare thee well, my love, my fatherland!}
\begin{verse}[\versewidth]
Extend to me your hand, \\
Another grievous hurt, \\
And yet another German heart; \\
Then fare thee well, my love, my fatherland!
\end{verse}
Oh, how sweetly the violets scent . . .
\end{document}