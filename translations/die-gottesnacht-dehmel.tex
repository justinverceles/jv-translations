\documentclass[12pt,a4paper]{article}
\usepackage[english]{babel}
\usepackage{microtype}
\usepackage[a4paper,margin=2in]{geometry}
\usepackage[T1]{fontenc}
\usepackage{fontspec}
\usepackage{ebgaramond}
\usepackage{dashrule}
\usepackage{verse}
\tolerance=1400

\title{The Godly Night\\ \large{\textit{A somnial experience}}}
\author{by Richard Dehmel\\translated from the German by Justin Verceles}
\date{}

\begin{document}

\maketitle

\noindent\makebox[\textwidth]{\scshape{first dream}}\vspace{10pt}

I was about to fall asleep—I could feel it. And I really hoped so for respite from the somber thoughts that had been rumbling in me for hours since the evening, when I’d received the death message. I pondered about the young woman’s willfulness to have chosen the slowest suicide method; but I was already delivered from the meaning in the words that shot through my weary brain. I listened blissfully to the throstle’s call, which sounded like the words “throttles all,” and I wondered at the images evoked by each phrase. Then at once it stood clearly before me: the mysterious mannequin.

Just how had it gotten in my room? There it stood waxen-faced between the door and cabinet as though risen from the dead. Its large, glassy, golden-brown eyes stared so familiarly into my heart, as though they’d watched over my play in early childhood. And there was a softness in their expression, as though they were living—as though they loved me—almost motherly. But of course, it only seemed so; I only had to properly remember. Indeed—my mother had gifted it to my children for Christmas, this life-size mannequin; and its smile stayed so immobile around its narrow lips, as did its stiff brocade coat’s folds around its gently curved shoulders. Yes, it was dead—dead like the fine, fantastic flowers of the old Indian templar coat that enveloped it down to its feet. At one time, I’d played between such flowers and plucked a bouquet from them—for their pale folded fingers. I’d still worshipped them then. For they sat enthroned on a gilded, ruby- and pearl-adorned altar and were the goddess of charity; that must’ve been many hundreds of years ago. Why did she stare so stiffly into my heart now, as though I’d killed her? She’d killed herself! Maybe I was dreaming?

No, she continued holding her hands with folded fingers and stood large between the door and cabinet. Maybe she would move if I prayed with her? She used to move after all; when I touched her joints, her burst wires rattled, down into her hollow chest. I sighed, and they rattled again; and her arms twitched a little. What if she never touched me again? And just kept looking at me so fixedly? I felt a pang in my chest, as though electric sparks were flickering over from the wires. Again I heard the quiet rattle; or was the throstle still singing? I wanted to stretch my hands out entreatingly, but the pang in my chest urged on into my fingertips. I wanted to turn away my gaze—but her gaze follows me.

It’s just a dream! So I try to convince myself; but she looks at my hands. At the ruby ring on my left hand—it begins to glow like an altar-light. At the wedding ring on my right hand—it begins to glisten like pearls of tears. And at the ring my father gifted me when I was still unwedded. Why do you torment me, Mother? So I want to groan; but her gaze seals my lips. I want to erect myself; I lie spellbound.

Her eyes begin to shine softly, and the rings’ glister grows more sparkly. Her eyes, too, sparkle desirously; at once the light expires. Those aren’t my mother’s eyes! My mother has a gentle regard, my mother is pious! And no longer are they the clear, golden eyes I once worshipped because it favors the regard of my children’s mother. These eyes are black—no—dark-gray, and know neither faith nor piety; they are the suicider’s eyes. Why did you have to kill yourself? So I want to ask her, and to my horror hear: “You willed it yourself, my beloved!”

I want to deny it and see her smile. Maybe she didn’t say the words at all. Or maybe I didn’t understand their meaning; her language has always been so ambiguous. But she hangs her head so peculiarly. Oh, yes—I wanted to throttle her. I hear the roundelay again, coming from the woods of my homeland. Soon my father will appear amid the trees. No, it’s the distant sound of a flute. No—a violin’s anxious exultation. My dead friend used to play like that, when we would still run childishly through the heather and look for wood-fairies behind the birches. Oh, he wanted to become king of the fiddlers, and here he comes sorrowfully in the queen’s wake. The hunters halt at the forest’s edge; and we all genuflect before her.

Why does she look at us with such scrutinizing silvery-gray eyes? That’s not my friend, that’s me—and Queen Elizabeth beckons me. “Rise, Shakespeare!” she whispers; and I feel us straighten up. He’s still wearing the black scholar’s attire he wore when he ran away from school, and a mad old brocade hat with yellow parrot’s wings. For as I know, we must play insane before the faithless queen. For she regarded him desirously yesterday when I declaimed against \textit{Venus and Adonis} at the hunting guests’ banquet; but he loves her chamberlady with eyes like a goddess, like a wood-fairy, or a deer. It peers in mortal fear through the barrel, and I stand and stare at it like a bloodhound. Oh, how very well we play mad when we love none but a goddess and wear such insane hats! And now she suspects why he became an actor and poetized poor Hamlet; and we brandish the hat before the faithless queen, and she smiles charitably.

She smiles ever more charitably; it pierces us through breast and brain. I want to throw the hat before her feet, and I do it, and stand frozen—the hat has black thrush’s wings and flies back onto my head. Her smile turns so dreadfully charitable that I should like to slay her for it. “You’ve done so already, my beloved!” she whispers to me motionlessly. That’s not true! So I want to groan; but is that still the English queen or just the Indian mannequin? If she stands there by the door any longer, she’ll really make me insane! Why does she torment poor Hamlet so? She’s his natural mother! She has piercing eyes like a goddess’s which penetrate my brain! Suppose God is just some gruesome shrew? In perpetual pupation?! Mother of All! But she has burst wires and hangs her head so peculiarly! I don’t believe in deities anymore! So I grate out through my paralyzed mouth. And with an immense triumphal feeling, my soul knows: it’s just a dream!

If only the wires would cease their endless rattling! Really it’s remarkable. They rattle louder and louder—as loud as the little old organ in my home town’s church. I read the golden numbers “1641” on the black-laquered plaque between the eleven apostle portraits; naturally, faithless Judas has been excluded—even as a kid I comprehended that. “\textit{Salvator Mundi},” it reads under the twelfth portrait, on a clear, heavenly blue backdrop; and beside the iron-fitted door sits enthroned the mother with her infant and smiles. I hear the organ sound its laud and weep for Christmas bliss. The altar cloth’s silver fringes swim in my pearly tears. I play with these pretty pearls, and Mother watches me and smiles. I am a child again on her lap and am no longer bewildered whatsoever. I’m just a bit inwardly surprised: the Apostle Thomas has three hands. Two are smaller, well-kempt ones; but out of his brown-red cloak reaches a third large leprous one. It grasps a book and I’m appalled. But I mustn’t even stir, lest it reach for me. I stare at the book. Suppose books can fall ill—

—and suddenly I breathe a sigh of relief: I realize it’s not a hand at all but a fold in the cloak, lying shoved aside over the book. I want to move the fold, but I mustn’t, lest the sacristan come to take the book and strike me round the ears with it. Already my ears are pounding; increasingly he pounds my ears. With bell strokes? They resound like thunder through my brain. No, lightning is striking around me; oh, Heaven, help before it strikes me! I want to hide; oh, Mother, where are you?! A light beam dazzles and shuts my eyes; I am struck, the beam is tearing me apart. An immeasurable whirl of colors sparkles from my head and sprays the heavens. I cry in ecstasy: “My magnificent brain!” And a voice replies from above: it has splattered up over the stars. I want to go toward it—oh, heavenly light! It shines in my eye; I awake.

On my nightstand burned the candle by which I’d leafed through Shakespeare’s sonnets to still my thoughts; and on the wall, between the door and cupboard, flashed the mirror’s edge over the portrait of my mother. I closed the book and snuffed out the candle. 
\settowidth{\versewidth}{What wake in sleep—what staring eyes.}
\begin{verse}[\versewidth]
I shouldn’t to a flame confess \\ 
What fervid gazes burn in men. \\
No mirror shall e’er clarify \\
What wake in sleep—what staring eyes. \\
Between dark walls, I shake and sense, \\
Behind each soul, the countless rest \\
To which I’ve not a gleam to hold; \\
The light shall yet unite us souls \\
Where all in quietude suspend, \\
All at easy rest . . . 
\end{verse}

\pagebreak\noindent\makebox[\textwidth]{\scshape{second dream}}\vspace{10pt}

We descended the hill’s tree-root steps, a company of ten, twelve men; the forester’s office lay above, buried in snow. All were silenced by the bitter cold; the snow swallowed the sound of our footsteps. The dachshunds followed heedfully after us, hobbling for frost in the downtrodden path. The morning sun poked through the rough, rimy birch twigs. The stiff, needly beards of the pine plantation bristled from their white coats. A badger was to be unearthed—I don’t know why. The good Lord came to mind.

Suddenly the dogs started barking; there came the clatter of wheels. The old village kneader came veering around the corner from a hideaway path, perched on her dogcart; it was drawn by a gun-shy cur of a hunting dog, the runt of a neighbor forester’s litter. Our little dachshunds began yelping. The long-legged one is at a loss for what to say; his tail tucked in, he sets into a trot. Their barking livens up. He catches on; and halloo, all tails raise and the wild hunt scatters, snow-blown, baying and bellowing down the path—the wrong direction for the good old woman sitting and swearing and wailing on the jolting wagon, both arms clasped around her pannier. We, laughing after her with long leaps—finally we catch up down by the railroad embankment. The dachshunds huddle up ashamedly to their masters; we repay the old lady. And again I think of the good Lord.

We tread on, damp with sweat. The snow grows blinding and strains our eyes; the rails glimmer. Over from the other side, a shotgun barrel emerges sparkly over the embankment, and a familiar otter-fur cap. “The neighbor forester,” says someone shyly; one turns as pale as the snow. Now the old one stands erect up above, in green gala, his bare red fist on the crown of his hunting dagger. His gray goatee pearls with ice, his big hook-nose casts a shadow over his cheeks’ furrows up to the ear; his dark-blue eyes burn with vigilance. “Come here!” he calls hoarsely. The blanched one obeys; now they stand on the middle of the embankment, in the piercing light. “Remove your gloves!” I hear, to my terror, feeling how the old man restrains himself. “Where is the ring?” he asks threateningly. He trembles. His fingers tauten around his dagger’s handle. A jerk—the edge flashes. Till halfway—and laughing derisively, he pulls it back. He spits at the snow with ineffable contempt, turning to leave. “Father!” I cry out, falling to my knees. He leaves.

I’m seized by a spasm. The fixed orbs of my eyes watch me twitch—from afar. Heavy, pointed tufts, the peaks of pines, swish and lash against my forehead. They transform. Kermes oak branches rush around me; I see the red berries tear long curves through the gray web of my breath. But over and over, a soft hand brushes its tender fingers through my hair. My teeth unclench; I believe I’m becoming someone else. He lies at her feet, his head held in her lap. “You’re still alive?” he asks with wonder. She slides in the armchair; the faraway red of the vernal evening gilds her bright brown braids. Beside her, on my writing desk, stands a fine Venetian chalice, purple-hued, a lily’s calyx, trickled through with gold, and a shimmering sea-green snakelet wreathes up around it. A kermes oak leaf stares out the calyx, and a waxen-like hyacinth—which she’s brought to me now; the lush blossom befuddles me.

“Give me the ring!” she coos. “I can’t,” he pleads arduously; and I hear him recount with an anxious voice the story of the ring. After the Battle of Torgau, his great-great-grandfather, the hussar sentry, was rewarded it for his courage and lasting loyalty, from old Zieten’s own hand—maybe even from great Frederick himself. He regarded the king’s pressed-iron portrait in the thin golden hoop: “And the eldest son always inherits it.” I hear his words as in a dream; it’s as though I were reading them from a book. “Give me the ring!” she coos. He wavers. “Do you feel guilty?” she whispers; “You—?”

What! Is she trying to mock me? I press my teeth menacingly against her knuckles. She smiles, takes me from her knee, and holds the hyacinth to my lips. I drink in the scent and remember. “You already have it,” I reply and look down at her fingers. “There’s still the other,” she coos; “The ring of the other!” Her gray eyes grow ever more bewitching.

I feel a violent tremor; I would like best to strangle her again. Then I could be faithful to the other, who bore my children. My gazes fix addle-heartedly upon the ruby on my left hand; it pearls like blood from a fresh wound. “Conscience is the spook of our dead God,” at once she gives voice to my thoughts, into my ear. I don’t know if she means it derisively. I want to explain it to her; she rises. “You’re too good,” she breathes ghostlily—“Only good men have a bad conscience; I never had one”—and pulls the ring from my finger. I want to resist; she floats away. I want to rush after her, to no avail; my knees writhe spellbound on the floor. I try to find the word that will free me.

I stammer verses, long, pleading lines; she fades ever farther out into the night. I see her disappear, pale as a phantom; only the ruby is still glowing like blood in the moonlight. No, like a scar—my dead friend! He floats over with his violin. He begins playing along to my verses—distant, pleading tones, from a soul that’s turned untrue. The round wound on his forehead opens up; blood-drops pearl from the little opening, with every bow stroke, down his pale temples, into the snow. The ruddy trail hovers nearer and nearer; his closed eyelids twitch, paler than his shroud, and I look for the word, the word—we knew it as children.

He opens his eyes, his bow stops, and I’m struck with horror—those aren’t his eyes! It’s—“the other”! My gazes grow weak, my mouth gives out; my fingers squirm to touch her garb—help me! The word! She points to my rigid body—long chains of verses, like banners, garrote my wrenched throat. I read and read with fear:
\settowidth{\versewidth}{Empty noose . . . your discernment—}
\begin{verse}[\versewidth]
Heavy rings . . . court . . . I court . . . \\
Empty noose . . . your discernment— \\
Dusky room . . . ancient heir . . . \\
Error . . . gloom . . . an emergence— 
\end{verse}

Who’s breaking the chains?! The door flies open. Beams of light dash against me like pinpricks. My mother stands at the threshold; she looks at me with an ineffable sorrow. My arms strive toward her—to no avail. “Sinning against the mother of your children?!” wrestles from her lips. Mother! So I want to plead to her; she refuses me. “God!” wrestles from my lips, aloud, the word—I awake.

A bright narrow moonbeam fell through the dark room onto my bed; it twitched. I looked to the window; there was no gap. I averted my gaze; the beam slid along. I don’t know what kind of light twitches like that.

\settowidth{\versewidth}{There appears from out the lifeless chamber}
\begin{verse}[\versewidth]
Come midnight, \\
When between your sleeps, \\
Your racing, beating heart \\
Has chased you from your dreams \\
—your timid, bated breath— \\
And through the dusky room, \\
Before your eyes, fly pallid shadows \\
—have you felt this fear?— \\
There appears from out the lifeless chamber \\
A beloved face \\
With unmoving eyes \\
And a silent wish for life \\
—do you know this fear?— \\
For justice, \\
With your own two hands, \\
You’ll seize your throat and throttle it . . . 
\end{verse}

\vspace{20pt}\noindent\makebox[\textwidth]{\scshape{third dream}}\vspace{10pt}

Maybe I killed her after all. Why should it be impossible? I did once kill even a child—an innocent, lovely little child. And I still believed in God then, in Hell, and the Last Judgment. I was a Swedish cuirassier with the accursed German Protestants and we pillaged a Catholic parish village. Oh, I can still feel that heavenly bloodlust, those struggling peasant women we’d locked in the spinning house. Then I just skewered the unruliest one’s screaming child away from her arms and cast it up and out into the village pond. I can still see clearly the little hand sticking out the marshy water as we later left the countrywomen; all coated in curdled blood, the child stuck out amid the rushes like a thick red tulip. But I have no dread of it; none else knows that I did it. Only I shouldn’t give myself away if they ever bring me to trial.

If I could just remember which of the two I killed. It couldn’t have been the mother of my children, could it? She’s always forgiven me. But of course the other one killed herself; her hand can’t testify against me. In any case I must attend the funeral; otherwise people could cast suspicion on me. And I must lay a bouquet on her coffin—a big, heavy tulip bouquet, so her hand can’t stick out. But they must be white tulips; the red ones have such a strong fragrance—it’s the pure scent of a decaying corpse. Why is the florist looking at me like that? With regular gravedigger’s eyes! But I don’t want white tulips! They look even more corpse-like! He laughs; I leave the shop in a hurry.

The street is bright with such a pale light as I’ve never experienced before. I can hardly carry myself in this light, with so worldly a weight it hangs around my head. There’s not one person walking on the pale street, and the houses are built as though from shadows. If I didn’t know where I was, I could believe in a spirit realm. But it enfeebles me, this light; it’s as though it’s trying to ring me out; no soul in the world shall enter my soul. But then I must retain my strength—my body is already as good as hollowed. Oh, but I’ll get hungry; I haven’t eaten anything today.

I don a harmless expression and enter a butcher’s shop. His wife looks at me questioningly—stock-still and questioning—what is she looking at me for! “Give me this piece of pork sausage!” I say with a slow, steady voice, as though I weren’t hungry at all. Again she looks at me mutely, lays it on a white piece of paper, and hands me it over the counter. I want to take it but I can’t move: at once I realize that it’s not a sausage but a little child’s hand, all covered in curdled blood. I stare distraughtly into the woman’s eyes: they’re the eyes of the peasant woman whom I’d trespassed against long ago. Finally I pull myself together and stagger out; behind me I hear a dull laugh.

I stagger on as if through mist and encounter a breakfast hall. Sitting and eating there behind the great windowpane are about a hundred men; no one will notice me there. I sit down over in the shadows and order something quick. It’s so loud in the dim room, I can hardly understand my own words. The waitress brings me fresh lobster and amiably wishes me a pleasant meal. It’s actually delightful how good it smells; but what’s she standing and waiting for! But I shouldn’t let anything show; maybe she just wants her money now. I pay her; she still keeps standing. It’s getting difficult not to yell at her; but I nod to her silently and quickly reach for my dish. I want to gently break off a claw—but what’s that? Just what is that?! I feel my face blanch into my lips: it’s a little red hand, and I’m struck with the smell of a corpse. And everyone’s looking at me; a hundred pairs of people’s eyes watch me inavertibly. And they all sit as still as spirits; not a sound is left in the dim room. I grope my way arduously to the door and into the open air; a roaring laughter resounds after me.

Just where can I get something to eat! If I walk around silently any longer, I’ll faint from hunger. It’s not that my secret is strangling me; only I’m prickled more and more sharply by pictures of the most splendid fare. Wait—I’ll visit the painter again, who’s always a delightful time; he’ll take my mind off things. I see him painting fruit; a great golden pineapple stands in the malachite-green bowl, and a few red tomatoes lie beside. “Can I have a tomato?” I ask frankly; “Tomatoes are my favorite.” He keeps painting in silence; why is he so silent? Then I see it. “Don’t make jokes like that!” I suddenly stammer—he’s painting a red child’s hand. “Don’t laugh!” I restrain myself; “Tomatoes really are my favorite!” But he doesn’t laugh at all. He just smiles—just looks peculiarly into my eyes and says with a sympathetic voice, “I think you’ve wandered in the wrong door; the courtroom is next door.”

For a moment, it’s as though I’m in a dream; it’s only as if through mist again that I feel the painter gently lay his arm around me, guide my staggering steps, and close the court’s door behind me. I want to awake from this dream; but now I seem to remember perfectly that in reality I didn’t kill anyone—neither the one nor the other; but is that really the reality? I’ve awaked from several dreams only to find I was still dreaming. I’ll rather compose myself so I betray nothing of my secret—not with a word, and not with an expression. I look at my judge.

Suppose it’s a Vehmic court? They sit opposite me, unmoving—eleven quiet figures shrouded in black, with eyeholes in their hoods. But no eyes glint in them; staring at me from the black masks are even darker holes. Maybe they’re only shadows? Sitting in hollow garments? Maybe there are spirits after all? For in the middle sits a maskless one, with closed eyes like a dead man, with silvery white hair on his head and face, and with an eternally imperious brow; I’ve often felt dread before such a brow. I don’t know—is it my father’s brow? Or the brow of a god? If it’s only spirits before me, then mustn’t there also be a superior spirit?! Could I just see his eyes! Maybe they are my father’s eyes—my father’s glorious, steely blue eyes, which often beamed at me with such hard fury, and yet as soft as embers in the fieriest fury, and then with such teasing, warm forbearance. But now he sits there so cold and stiff, as though he’ll never open his eyes to his son again—unless I open my conscience to him. They all sit so cold and stiff, as though they would wait an eternity for it. I can feel that I should finally speak.

“Your Honors!” I begin dauntlessly; “Truly, my conscience is clear. Even supposing I did kill them, in the end they killed themselves. For the one really did kill herself, and she made herself kill herself. And since she had no conscience, she took mine and then couldn’t suffer it. And the other one, to whom my conscience belongs, and who thus always forgave me—otherwise I wouldn’t have had it taken from me—she couldn’t forgive it any longer. For since I didn’t have a conscience anymore, and if because of that—which I don’t know—she perished for sorrow, then essentially she perished by, in, and of herself. For even if I’d willed it, essentially, your Honors, it was willed by another. For if I stand here before you all—and if, as I can see, my father is now God—essentially, I’m my father’s son, and my will was God’s will. So if I, your Honors—no, not I, if I’m the son of God—so if God, your Honors, killed one of the two—no, the other—no, both—no—all the others—”

Suddenly I stop and can only stutter; I realize I’ve bewildered myself. I try to read my judges’ looks and see only black holes. Helplessly I look at the one sitting gloriously in the middle and feel dread before his clear brow; at once I catch a vague memory, as though I’ve killed countless souls from time immemorial. And finally God opens his eyes to me—my father’s beaming blue eyes open of eternal repose, and he asks my soul: “Do you plead guilty?” I hear my heart in his voice and see my life in his eyes. I know I just have to say no and I’ll be acquitted forever. I feel the no on my lips; I need only open my mouth and I’ll be delivered from all travail. And I open it—“Yes.”

Terror strikes me like a blow. I feel my consciousness numbly dwindle away; it feels as though I’m plummeting endlessly downward through innumerable dark, bottomless spaces. Or am I plummeting endlessly upward? I hear singing voices from above—are they human voices? Spirits’ voices? They sing me back to my senses—two women’s voices sing from afar, from whence—from whence?—from whence thy dreams! Free thee, soul—from time, from space!—they fade away. Slowly I open my eyes; I see myself walking through an arched gateway.

The light still has such a worldly weight, like nothing I’ve ever experienced—a vernal light, as yellow as death. Only, there’s a man striding before me in a judicial black gown, and I must heed his footfall—then the heavy light will lighten. They sound strangely familiar to me, these steps. I must’ve heard them often and followed them, step by step, as I now try to keep step with them amid the resounding arched hall. Is it my father? My heart says no. And then I hear the same steps behind me—only vaguer and more unstable. I turn and stand astonished; and the man before me turns too. I see that behind me walks a young man, and he’s me from years ago; I see that before me stands the man who I’ll be in the future. He waves to me briefly, and it wafts his gown, and we stride in step out the gate. And it wafts his gown, and soundlessly the man strides out of himself and vanishes from my mesmerized gaze. For my regard clings to a fatherly, eternally imperious graybeard, remaining in that one’s place, and he beckons me to keep following him. So we come to a harbor’s waters.

The water seems to extend immeasurably under the deathly yellow sky. Many great ships are docked on it, with tall masts adorned richly with pennants; but the yellow weighs down as heavy as night, so that no other colors may dawn. Everything—the ships, the pennants, the water—appears as shadowy-black as my hoary guide’s gown; only his white hair shimmers silvery in the twilight. What sort of ships are those? So I ask myself dubiously. “Real ships”—he replies tonelessly and indicates a dock in the western sky. Not a sound of work can be heard coming from the shipyards; the whole harbor seems deserted. The black stakes around the slipways loom stiffly along the horizon like a bare, undead copse of giant antediluvian shrubs. Only from the copse’s western edge does something gray ascend clumpily and bestir itself in the heavy stillness; it stirs like the stony gray, ponderous head of an elephant. Is it the head of some ghostly idol? Is it a God’s holy crown? But my guide beckons me to look.

And what seemed like a head begins to brighten and emerges from the towering black copse as a great, glamorous moon. It doesn’t glisten so wanly as a nightly moon, nor so glarily as the daily sun; it glistens like an early-morning dewdrop, and all the colors clarify. And now my guide turns his grizzly, blue-eyed countenance toward the eastern sky, and with a slow, imperious hand evokes into the transfigured night a second such glamorous moon. “You know, you should believe in the might of spirits”—so he breathes into my shuddering heart and wafts away one of the moons. Am I blinded by his breath? At once I see only light. And only blindly do I then feel myself shine and hover into the boundless blue. I have a vague inkling that I myself am the graybeard; and he’s hovered over to the other moon? I hover too with outspread arms and abstracted eyes toward him.

The shine grows ever fierier; I breathe with rapture the gentle embers. I hear singing voices from above, two invisibly remote voices. Are they the souls of the two female spirits again? Awaiting me on the beaming moons? They sing me higher and higher aloft—far away, soul—from whence thy dreams!—awake at large—from time, from space!—they draw near. They draw so softly near like diffident winds; they kiss my outspread hands. In my palms rest their lips, my lifeblood flows to their kisses. They kiss ever more heartfully, and more spirits sing from above. Are they trying to kiss the life from me? “Free thee, soul,” they sing. Can they only live if I enliven them? “Awake thee, soul,” they fade away. I pull together all my heart’s might; a hollow horror gives a groan from within me. I want to wrest myself from the deadly kisses; like a man on a cross, I hang powerlessly. With my last strength, I bend my fingers, and while a heart-rending howl tears open my splendor-bathed eyes, I hear that the cry is my own, from which I’ve awoken, in tears.

I lay there truly like a cross-fixed man, arms outspread in the dark, palms stretched over my bed’s edges, rightward and leftward into the blackened air. I shoved my half-benumbed limbs slowly into another position and shut my eyes again. The restful darkness did me good after my mad spiritual conflagration. I decided that if I dreamed like that again, I would turn my thoughts immediately to my body.
\settowidth{\versewidth}{From whence, from whence?}
\begin{verse}[\versewidth]
Free thee, soul, \\
From space and time, \\
Awake at large, \\
From whence thy dreams; \\
From whence, from whence? \\
From time, from space, \\
Which e’er remain, \\
Awake thee, soul, \\
And drive away, \\
From hence, from hence, \\
Your dreams afar! 
\end{verse}
\vspace{20pt}\noindent\makebox[\textwidth]{\scshape{fourth dream}}\vspace{10pt}

But I must attend the funeral. Or at least visit their graves. For by now they’ve already long been buried; indeed I was at their cremation. If only I could find the right burial chamber! I must’ve gotten myself lost down here. Where is that vault of urns! There are nothing but chambers of skulls here. And the passages in between are so poorly lit, one loses his every sense of direction. When I get back to the upper graveyard, I’ll have the board set up better signposts. But how do I get back up anyway! I remember reading that there have already been people who have lost their lives in these convoluted catacombs.

Just where is the light in these skull-chambers coming from? It’s not an electrical installation; there must be some sort of skylight. Whence the dull, dim, underground glimmer in the intervening passages. I’ll stop looking left and right and just keep following the passage straight ahead, toward that strange bright opening. It stands like a white box in the dusk; there must be a door to the outside. It appears to be brightening gradually; already it’s almost blinding me. But the white can’t be the white of the air; it stands too immobile, like a stone. Its edges are so brightly defined, I must close my eyes. But I keep going straight toward it; I feel myself step through it. Suddenly I can breathe much more easily; so it must’ve been an exit after all. I open my eyes and see: high above me yawns the blue expanse.

I look and look: high above me—and above four high, bright white walls ascended vertically around me. Shall I never then find my way out of this senseless labyrinth? But I shouldn’t lose my composure; I’ve long known from experience to just turn my thoughts to my body, and then the soul will come to its senses. So first I’ll look at the room to see whether there isn’t an ascent somewhere. The room has four smooth walls, bright like crystals, consisting entirely of quadratic fields. In the middle of each field is a gold star, ingrained delightfully into the crystal; but nowhere is a foothold from which to get up. It’s a wide empty hall; it seems to be nothing but some sort of air shaft. But look—it has another door, directly opposite the one I entered through. And there on the edge is a handle to which a string leads from the passageways; that must be a guideline. I grab the line to move along and look back one last time.

But what’s this? Have I really lost my senses then? The other door too has such a handle, to which leads such a guideline. I must’ve overlooked it in my search through the dusky passageways. But the doors have the same exact form, and I’ve been turning in the empty hall—which door did I enter through? I touch the line and myself; it’s all completely material. So I can move along calmly; if I’m careful in my search, the right direction will reveal itself. I touch along the string, occasionally brushing against a handle; and again I’m in chambers of skulls. But the light looks paler here; and the passage seems to be sinking gradually deeper. This light can’t be coming from above; it seems to be collected from the bowels of the earth. The skulls all glisten as bright white as the empty hall’s crystal quadrate, yet all around me are deep shadows. And each of these skulls was once haunted by a world—with gold stars in it, and blue skies, and maybe even an eternal God; I feel an insane desire to seek God in these skulls. But I don’t let go of the line; I don’t want to lose my direction again.

Now come chambers of animal skulls, too; they shimmer just as chthonically. What’s that stirring in the shadows? Is it possible—my old faithful companion?! Come, my dachshund, what are you looking for! What are you looking at me so interiorly for? Indeed, I killed you; but why did you always growl when my late lady wanted to kiss me! I had no choice but to poison you! He just keeps looking at me soulfully, the same gaze he gave me as he lay before me in death throes—entirely without reproach, entirely devoted. But what does he want now—he’s alive! Maybe he wants to lure me into the chambers? I grip the guideline faster and recall my body; I’ll just keep walking along, of course the dog is nothing but a haunt.

No, he’s following me; I hear him behind me. I stop; he stops too. I turn around; he lies down. I try again to call him; he doesn’t move. He just keeps looking at me with his endlessly devoted eyes; and no sooner do I begin walking again does he follow me, step by step. I hear his quiet paws; I feel his gaze cling to me. Entirely without revenge, entirely full of love—as though I were being followed by the good Lord. How this godly gaze chases and torments me! If he keeps clinging to me, I’ll kill him a second time! But I mustn’t lose the guideline; otherwise I’ll finally perish myself, in this demented labyrinth. Wait—isn’t that another light shimmering ahead? At last, that must be the columbarium. Indeed, the square grows brighter, and the line appears to lead straight toward it. If I could only move faster; the gaze behind me bears down on me like a gravestone! I force my feet to run. I gasp for the shining hall. I ignore the pain in my eyes. I almost stumble into the blinding quadrate—and through! And I recoil in horror: I’m standing again in the crystal room, the open sky above me—I’ve wandered in a circle.

And what’s that groaning, what’s stirring beside me? The dachshund comes creeping after me through the door! I see clearly now it’s only a shadow—a mild-eyed shadow. I rush at him with racing hatred; I’ll tear the spook apart! I seize him with both hands, by the nape and lower back, and pull and pull. He writhes in my grip; he stretches to and fro like rubber. I feel in despair how he cripples me—how easily and pliably he emasculates my arms. I feel down into my inmost flesh and soul: if I can’t overcome this specter, I’ll be powerless for all eternity! I strain all the strength of my nerves, even if my brain and veins should burst! And a jolt, a quiet dying whimper—oh, bliss, I’ve rent the shadow. With one last devoted look, he dissolves into thin air.

I stand and tremble with my whole body, for happiness and weariness and renewed desperation. I stare up into the blue sky—is there no escape from this crystalline grave? I feel my exhausted limbs—why should I keep my thoughts on my body! It’s not necessary now; who ever talked me into that in the first place? How soundly I could sleep in this noiseless shaft. I’m so tired. I hear the song of my soul. Is that the sound of wings rushing in the azure? No, I don’t think so; I don’t see anything. Except—a white feather floating down. It comes whirling like a snowflake. And another, and another, fluff by fluff—right down into the middle of the hall. More and more and more fluffy white feathery flakes; already the whole floor is covered. I have to step back to the wall’s surface; there’s already a hill, soon a mountain. Oh, holy salvation—the mountain grows higher and higher! Now it’s nearly as high as the shaft’s edge, and the fluffy swarm piles up ever more thickly. I leap in with both feet; I sink into the downy flood. But it conglomerates under me; I stamp and stamp, and I’m lifted higher and higher up, as though balls were bouncing me aloft. I can hardly see, so it sprays around me; and burning sweat seals up my eyes.

There—a fresh puff of wind cools my forehead. I feel delighted—I’m up, I’m up! My eyes venture ajar again, squinting through my damp, fluff-veiled lashes. No more feathers spray around me, and the heavens shine blue; it’s a celestial stillness. I stand on a steep, staggering summit; deep below me gapes open the white abyss of the labyrinthine shaft. Oh soul, soul, how do I cross over?! Look—all around the well, like a Garden of Eden, lies the blooming, verdant graveyard! And the soul sounds: I see it, oh spirit! I see it through tears, oh divine spirit, through my iridescent tears! Yes, the summit sways, and a wind comes rushing, and you, swaying one, weep and I spread my arms—if you, oh godly spirit, but answer my soulish self now, I will trust your might eternally!

Hark—oh soul, doesn’t the rushing wind give wings? And the summit dissolves and floats and becomes a cloud! See—with both arms, I embrace it and float out over the chasm. Oh, how soft a flight in this lightsome fluff—I feel neither heights nor depths anymore. I feel only the rocking of this windy cloud and the sweet tickles and pricks of every force. Is it trying to rock the life from me? Then know this, soul: my flesh laughs! I can release it whenever I wish; I’m fledged all over! I can fly with you wherever I want; I need only blow away the fluff! I blow and blow—what’s that? I’m blowing into my own nose! Am I myself the tickling wind’s maker? I sneeze, I laugh, laugh—and awake.

I still lay in the dark bed, and I held my pillow in my arms. I felt a small feather stick out the crumpled cushion; it touched the tip of my nose. I removed the feather and laid the pillow flat; I hoped to sleep just another hour. The morning already appeared to dawn; but I was still tired enough.
\settowidth{\versewidth}{There should hang a God’s hearty attendance—}
\begin{verse}[\versewidth]
If over our deepest despairs, \\
Where for nothing but opened-up doors, \\
We know neither an in nor an out, \\
Should be watching an unturning eye— \\
If among our highest delights \\
Where with trembling tread \\
And triumphal reserve, \\
We suppose to have trampled all threat, \\
There should linger His all-hearing ear— \\
If amid our loftiest nonchalances, \\
Where with eagle-like ease, \\
We suppose all pursual \\
Of death as of life \\
To be vanished in air, \\
There should hang a God’s hearty attendance— \\
I think He would perish of laughter . . .
\end{verse}
\vspace{20pt}\noindent\makebox[\textwidth]{\scshape{fifth dream}}\vspace{10pt}

Aye, my pursuers, you make me laugh! For I can fly when I wish; by my own force of will, I can fly! They race after me like chevied game, a pack of raving hunters and hounds. But here—I’ll just throw on my cloak and so escape their lunacy. Already I’m floating over the oaks’ tops, and I laugh a halloo down to them. I hear them roar, “You murderer!” all hoping to murder me themselves. They’ve all stripped naked, and still I’m faster than them. How they stare after me, panting for vengeance, pale-faced through the barren oaks, while I dwindle higher and higher away! Halloo hallelujah! So I laugh and salute to them—you might’ve uprisen at Judgment Day, but I’ll fly into immortality!

How they shrink smaller and smaller, their pale, shock-stricken bodies—they swarm like worms down among the withered brown foliage, like dug-up beetle grubs. I let my cloak fall outspread to cover up their wretched nudity. It floats down heavily, for I float up; with swimming arms, I part the clouds. What’s that glistening out the steely-blue ether? Is it an unknown star? Halloo hallelujah, cheers my discerning heart: it’s a world-illuming brow! Hail, path-wise poacher, you huntsman of the wicked, you! Shakespeare in the highest! His slumbering eyes stay shut; only a dreamy smile appears on his ghostly head, greeting me silently and illumining my course. Several other slumbering spirits greet me with starrily shining brows and illumine my sublime course. Rembrandt greets me, and Leonardo, as do Dante and Goethe, Beethoven, Bach. My father greets me, and my mother; and from afar shines a thorn-crowned head.

Where have I seen this stirring visage before? This forgiving expression that transfigures pain—was it in my childhood? I’d like to pass this visage now; but behind it everything is black. Nonetheless I’d like to float past it; but it draws me nearer and nearer. It draws me in with its circlet of thorns, which shines even brighter than the slumbering brow. It shines like a big branchy nest; the branches grow ever larger into the distance. I’d like to circle around this growing nest of light; but it stretches circularly around me. It whirls me high like a spark into that black immensity. I look down to oversee it: I see an immense brightness. I see a boundless hovering realm of light: a deep, whirling, resting nest of countless circling starry rows, endlessly branching out through the black expanse. I catch a chill of dread when I realize: I’m in another world.

The dread drifts more intimately, and it beatifies; I can feel that it wants to drift me to rest. It drifts me down onto that slumbering head—who are you, dreamful spirit, whose head cradles me in a realm of light? I move myself keenly down to the shining middle point of the crown; I sink with the bright bliss of a home-comer, ever deeper into the worldwide nest. And what appeared to be a point is really a vault, a starry, milk-white, neverending dome, on whose surface rests the nest-like circlet. I marvel down into the stilly, dreamy vaulted room, down through the shimmering dome of the head’s crown—you sublime head, I suppose that’ll be what we on earth call the Milky Way? Yes, I see them circle within you, the stars, the suns, and that earth, like blood cells in your veins, you resplendent, thorn-incoronate head! How they tremble, all the little souls, each thinking themselves worlds in their spheres—I see them tremble clearly in the mist, before your world-bounding brow. And all so far from my gaze, so boundlessly far, like that globe of the earth which I’ve escaped through the clouds into this other transfigured world. My eyes close for trepidation—who, who are you, transfiguring spirit?

A silvery laugh awakens me; was I dreaming or do people live here? No, a figure of light lingers before me; I jolt up, a female spirit floats around me. Did I know her on earth? Her eyes enliven my heart so familiarly, as though they’d watched over my play in early childhood. Her regard is of such an intimate silvery-gray—no, luminous black—no—a deep, hearty, clear gold, a wholly gilded, silver, heartfelt clarity; is it the goddess of charity? She smiles and lets her head hang a little—oh, sweet scapegrace of charity! She nods to me heartily again; I listen, and hear her soulful song.
\settowidth{\versewidth}{The earth a-sleeps in shining, shrouding haze;}
\begin{verse}[\versewidth]
The earth a-sleeps in shining, shrouding haze; \\
And yet its breath cannot its woe disguise. \\
It dreams it might for freedom wake; \\
Is it yet time to rise?!
\end{verse}
I stare down, and my trembling renews. No, pleads my gaze, let the earth’s soul rest! It lusts for revenge, it wants only to kill; let the spirit of this light-realm awaken us! And again, the spirit woman smiles—why? But her smile encourages me. Let us awaken him! So demands my gaze—Him, whose head bears this other world, yet under whose dreaming brow that earth still entrances us! Let his starry eyes shine, that’ll lift the spell!

She smiles and nods, and disappears; I grasp with bewilderment into the empty splendor. I float alone again into the distance; only her silvery laughter is still audible. No, her regard remains too; it hangs before me like a golden star, amid the silvery-white circling nest. Or—no—it’s a binary star! Yes, two shimmering twin stars, clear as gold! A whirling little pair of starry souls! Two little sparkling souls of starry cells striving to sprout together into one. I grasp at it, and shrink back: one of them is clearly reflecting my image. I see myself grasp up into the circlet, into the circling play of the starry branches—and in the other, isn’t that the image of the spirit woman? No, they’ve already both sprouted together—I don’t know if it’s my image or hers. It plays with the circling nidal balls of light, with countless whirling, endlessly small branchy, starry cells; and in all star-cells play such images. I want to grab it; I grasp into the intangible. I notice it float far above me, immeasurably far, and floating thus it sprouts farther, ever farther into the whirling plays of astral images; it only appears so small because it’s so boundlessly far away. It whirls me high, and already whirls from the crown, and sprouts away ever more awhirl above me, and a silvery laughter assails me on all sides.

I have to laugh along, I look down: floating below me, shrunken-up in the black depths is the worldwide nest of thorns, still but like a flat wattle, a saucer-shaped milk-white pane, spinning on which is an enormous swirling, spluttering, ever-growing whirligig of golden-clear star-cells, and it swings me swishingly along, higher and higher up the swelling edge; I can hardly guess at the tiny cellular proto-image in the spinning top’s point, playing ball down below with other such proto-images. I can only guess how from each image-beam that it splutters up into the silvery mist, a new flight of golden beams emerges, like pupae, from each worldly star a starry world, outbranching ever more enormously aloft, an unending fountain of limbs of luminous puppets, and from each limb an entire being, an entire worldly play of puppets’ limbs, causing other worldly limbed figures to emerge in all directions and play. I want to look upon one of these beings; I float so near its side, I can feel the rushing cycle of its breath. I want to discern whether it’s a man or a woman; but it expands its giant shining, misty body, which star after star whirls through like seeds in wind, so stormily into the immensity that again I can perceive nothing more than a soulfully roaring laughter. And I must laugh along with trepidation, for now in all my trepidation I suspect: maybe this immeasurable light display, this whole lofty whirligig of astral puppets, is just one small limb, maybe just the lowest tiptoe of an even greater figure, which in turn can play out still greater ones—oh, reveal yourself, most sublime being!

I stare up at the outermost luminous edge—if I could catch but a twinkle of his shimmering eyes! I endeavor to wheel more steeply up, to steer even nearer the maelstrom’s sphere of influence; I feel as though I’ve been doing it for eternity. I look back at my flight course; the starry nest below is no longer visible, there appears to be only the very lowest tip of this floating worldly whirligig. I feel so dwindlingly far of soul, I can hardly feel my movements anymore. And no longer can I see anything of my body in the growing soulish, lightful mist; it seems I’ve myself become a world of light. Oh if I could but espy the light of his eyes, what makes all these blissful mannequins laugh from their circles! At once I too must laugh blissfully: all of a sudden, within the spinning top, all around below me, from every corner of the mist, I see whole resplendent swarms of eye-lights—all those high slumbering spirits that at one time illumined my course, they awake from the depths of their dreams and follow me higher with laughing regards. There awake and laugh Rembrandt and Shakespeare, Cervantes and Swift, Aristophanes, Nietzsche. My father laughs too, and my mother, and that thorn-wreathed head. I want to greet it, but my greeting freezes—out of my gaze laughs the goddess of charity. I stare down from gaze to gaze: in all the swarming eyes, even in Your own stars Nietzsche, Rabelais, Shakespeare, you wild dreamers, you friends of the wicked, there plays the image of the goddess of charity. I feel giddy; I must look up again! Oh, awake, you most sublime being, awake from your indifference! Exalt me at last to your regard! Wrest from me all these watchful eyes—they remind me still of that long-gone earth, for eternity gone! Emerge at last—who are you, you—

I hear with fright—what’s that laughing, “You”?! And once again an echo laughs stormily, “You!” Is the sublimest being sneering at me? Oh, just bring me higher! I have no more trepidation! Just bring me! I steer even more abruptly into the spinning top, and laugh stormily along, “You, you, you!” I let myself be ripped wholly into the merry maelstrom—maybe even the sublimest being can only hear me from within him, there, there in the innermost axis! Yes, now I hear it laugh, “There, there, there”—and I see the whole worldly play of puppets begin to nod, wildly, far and near. And ever more wildly—my heart falters: does it want to nod me from my equilibrium? No, it nods in equally wild worldly circles, circularly under me, circularly over me—there, there, there—with the starry charitable eyes of a spirit—and laughs indifferently, “Ha, ha, ha!” Surely it wishes only to laugh me into safety; yes, the top’s axis is already very near—will it finally emerge there! Yes! All the spirits laugh a yes! And they nod. But what’s that? Ah! The axis! It spins us ever higher; but my heart falters ever more abruptly—it’s not losing its equilibrium, is it? No—is it contorting the light of its soul? Ha ha ha, its contorting our overview! It’s starting to wobble! Oh, all you spirits—it seems the sublimest being wants to stand on its head!

I hear with horror everything laugh again, “Yes!” Ha ha—halt! Charity! If we’re falling—we’re falling into the fathomless abyss! There, I see it—almighty heavens, yes: it’s standing on its head! It’s emerging! Ah—something stands high in the heavens above me: in the middle, from the wobbling worlds of souls, stands the crown of the whirligig in glory—and it’s—what?—a sole? Aye, a giant wobbly, worldly, soulish sole, around which wriggle countless tiptoes. I discern that it wants to wriggle us even higher—it shelters our world like an immeasurable falling cap, and we’re wriggling in an immense, world-sheltering, primeval puppet, which stands bottomlessly on the tip of its hat, and whose belly shakes from laughter. He shakes us too, along and along, ha ha ha! Halt, spirits, before he bursts! There—he bursts—I have to turn around for laughter. Ha ha ha, all the worldly spirits turn around too! Ha ha ha, they turn and contort my vision and hearing! Ha ha ha, the sublimest being exacts vengeance! Ha ha ha, it kills me with laughter—my eyes give a final twinkle—no, they open!—finally, they open—what? Am I awake?

Yes, I sat in bed with open eyes; and through the middle of my dusky room extended a bright golden beam of morning light, full of countless whirling sun-motes. So there was a gap in the window curtain after all. I rose, drew the curtains wide, and took thought in the brightness; then come evening I threw the death message from my Shakespeare and into the waste basket. I didn’t know if I should say a thankful morning prayer like a child or wish all of God, life, and the world at the devil. Even today I don’t know, you heavenly haunt, oh all-charitable fancy!
\settowidth{\versewidth}{And what wondrous, lustrous worlds abound within us!”}
\begin{verse}[\versewidth]
“Who are you?” “Whoever you wish!” \\
“Where do you live?” “Wherever you sense!” \\
“Do you rest in slants of sun?” \\
“Could be whirling swarms of dust! \\
Brush my cap, \\
Beat your smock, \\
Then you’ll find the one you’ve sought \\
And what wondrous, lustrous worlds abound within us!”
\end{verse}
\end{document}