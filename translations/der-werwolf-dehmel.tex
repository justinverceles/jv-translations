\documentclass[12pt,a4paper]{article}
\usepackage[english]{babel}
\usepackage{microtype}
\usepackage[a4paper,margin=2in]{geometry}
\usepackage[T1]{fontenc}
\usepackage{fontspec}
\usepackage{ebgaramond}
\tolerance=910

\title{The Werewolf\\ \large{\textit{A simple story}}}
\author{by Richard Dehmel\\translated from the German by Justin Verceles}
\date{}

\begin{document}

\maketitle

One foggy October evening in the remotest suburb of a great North German commercial city, the sinister news spread that the town apothecary had been shot dead by a robber on the railroad ride back from the city. It was around the same time that in a suburb of the German imperial capital Berlin, a released prisoner and journeyman shoemaker moved the whole newspaper-reading populace to unforgettable laughter by putting on a Prussian officer’s shabby uniform and concomitant posture and giving the assembled magistracy the mind-beguiling illusion—or as the cultivated Germans of the time expressed it, he \textit{suggerierte} to them—that he had been instructed most personally by His Majesty the Kaiser to clear out the official treasury. Even still on the day of the murder, amid all the then typical reverence for the state representative’s wisdom and dignity, there was abundant laughter over this so-called Captain of Köpenick’s exemplary swindle; but now the population, which chiefly comprised wealthy merchants and well-off officials, grew increasingly serious. Nearly all of them had to visit the city daily to attend to their business. So each of them said that they—the educated citizens of a civilized state, sat upon the taxed railcar upholstery and immersed in the well-deserved enjoyment of a newspaper or light nap—could have suffered the same exact fate as the piteous apothecary; indeed maybe it could still happen. For the murderee was buried without even the merest trace of the murderer having been found; and all weekend the city weapons dealers sold an astonishing amount of pocket revolvers, swordsticks, knuckledusters, and other defensive tools to the agitated populace of every neighboring locality, while the railroad administration received the most multifarious and urgent security proposals for renovating the entire fleet of cars, and the police headquarters the most manifold reports of suspicion, which contributed decreasingly to the murderer’s arrest and increasingly to the citizenry’s agitation.

Meanwhile, all that was ascertained was that on the suburban rail’s embankment, not far from the last stop, an old cavalry revolver with two discharged, two still-loaded, and two unloaded chambers indicated both the deed and the flight of the perpetrator; nearby the investigation officers also found the apothecary’s golden watch and chain, and in the railcar lay the murderee’s emptied and blood-stained wallet on the upholstery. Evidently, then, after methodically executing the robbery, the criminal had jumped out the moving car, slammed the door shut again, dropped the revolver in mid-air, and lost the watch too in his haste; or he had intentionally discarded both the watch and the revolver so as not to betray himself later. There were no footprints to be found anywhere on the embankment grass, and in the thick fog, the perpetrator could have very easily absconded that same evening onto the open road to the city harbor, cleaned himself along the way in some field pond, and then, presumably with false papers, taken to sea the next day aboard one of the many outward-bound ships as a coal trimmer or the like. From all the various reports, certainly, most local residents preferred deducing that he was still roaming furtively around the country; and as the massive distribution of pocket weapons of every sort naturally incited several brazen youths to new violent acts, general suspicion was placed repeatedly on the escaped robber, although these inexperienced, casual robbers always fell swiftly into the police’s hands. Otherwise all investigations—including appeals in newspapers and notices on columns inquiring whether anyone in the German \textit{Reich} had recently sold an old cavalry revolver—despite the offered reward, remained unsuccessful; alas, one had to conclude that the criminal might have set the weapon aside in his military days and reserved it for his later career.

What especially agitated the population and provided plenty of material for conversation was that the  apothecary, notwithstanding the one gunshot that bored through his temples and the other that smashed his skullcap, was found still living in the railcar, albeit already unconscious. The medical autopsy revealed that the unconsciousness had probably only occurred some minutes after the injury, and with severe pain; and everyone tried now to imagine what thoughts might have stormed through the poor wretch’s mangled brain in his last moments—and all the more intensively when, during the disembodied’s lifetime, he had more or less come near almost every single townsperson and was a widely admired personality: a mild, compliant, slender gentleman with a blond, Christlike head and—what seemed remarkable to many, considering also his enlightenment—an eminently religious devoutness. All the pensive people then posed to themselves and others the question of how well the apothecary’s faith might have persisted within him during the last short period of consciousness after this dreadful experience, especially as it had become known that when the widow first caught sight of the deceased, she produced only the despairing words: “There’s no God, there’s no God!” The fact, too, that the rather high sum of 150,000 marks, for which the scarcely forty-year-old man had insured his life only recently, and which the insurance company transferred to her immediately, was accepted by her with not a stir of solace—rather, she could barely sign the receipt through her sobbing—gave the sentimental citizenry cause for much sympathetic discussion. The population’s compassion and humanity extended so vastly that the cemetery gardener needed more than fourteen days to mend the graves and flowerbeds, which had been crushed and ruffled by the unremitting throng of mourners, local and nonlocal and of every age and standing. And still several weeks after the event, in the whole region, one couldn’t witness an educated conversation that didn’t finally lead to discussion over whether the immortalized apothecary, if there is continued life beyond the grave, would have adjudged his earthly killer’s undiscovered identity an entirely proper proceeding of heavenly justice.

Then one fine afternoon, a town greengrocer, preparing his hotbed for the winter, happened to overhear through his garden hedge a peculiar conversation between the little neighboring house’s owner and his only friend. This neighbor was a riddle to everyone. A one-time railroad conductor, a derailment had inflicted him with a minor head injury that, if his behavior wasn’t a deception, left him with a lasting mental derangement—not a proper lunacy but which according to the doctors nonetheless invalided him; and so he had received in court from the railroad administration his discharge, his appropriate compensation, and—until his mind might become fit for work again—his pension. Now from dawn to dusk, he did nothing more than stride grouchily to and fro before his wretched little house, for the acquisition of which he had dissipated his compensation. At every time of day or year, in all weathers, he marched up and down in the narrow space between his house’s wall and the street’s hedgerow like a wolf in a cage, with a wild, bushy reddish brown beard, both fists buried in his pockets, and cap pulled far over his face, and shyly surveyed passersby, sometimes with mistrustful, screwed-up eyes and sometimes with inimical, wide-open eyes; so that finally the townspeople said that if he wasn’t really mentally ill, with this sort of practice he was learning it to completion. Besides his meals and other domestic activities that his wife couldn’t perform for him, his public lifestyle only showed interruption when some death occurred in the neighborhood or even was only expected to occur. Then he disappeared immediately from the little street garden and locked himself daylong in his bedchamber or, during the funeral, as though haunted by evil spirits, ambled around in the heathland thicket that adjoined the graveyard. On this account, a town schoolteacher, who spent his leisure hours reading treatises on ghost stories and other horrific tales, once jested at the beer table that the mysterious red-bearded fellow would one day emerge a werewolf; and this offhand remark stuck as his nickname, insomuch that it became quite usual that no child dared venture into the heath alone for fear of being attacked and strangled by the wild man.

Whether the Werewolf himself noticed or suspected what was rumored about him, even his wife didn’t know; he had no tendency toward conversation but rather answered addresses with either nothing at all or at most a curt grumble. Only one toadish, humpbacked little mender with whom none else wished to associate had gotten close to him and understood how to occasionally extract a few words or even a grin from him, though this happened rarely enough and only on especially beautiful days; for the mender’s gout attacked his miserable bone structure at the slightest breeze, and moreover he was so weak on his legs that he could hardly keep step with the indefatigable Werewolf for even a half-hour. But when it did happen, his friend seemed to derive deep pleasure from watching how the pitiful little ball of misfortune with his beardless frogface and wheezy, yappy voice hobbled alongside him and how the people stole faraway glances at the strange pair. So one such fine afternoon—it was an unusually mild November—the aforementioned greengrocer, kneeling behind the garden hedge, heard the mender suddenly ask the Werewolf whether before his railroad employment he had been a sergeant or something. And when he answered warily that he couldn’t remember it all, the mender produced from his coat a newspaper page depicting the infamous cavalry revolver in life size and asked shrewd-faced whether he might remember this; whereupon the Werewolf stood still in astonishment and then burst into a terrible raging sob and probably would have sundered the cripple if his wife hadn’t rushed outdoors between them and if the greengrocer too hadn’t hurried to his rescue. Naturally the overhearer reported the incident forthwith to the police, and come morning two gendarmes conveyed the fiend to the city and stuck him in the detention center.

During the interrogation, the mender initially explained most self-effacingly that he had solemnly to deny any friendship with the arrested—that he was a respectable citizen and only associated with the suspicious man to study whether he was truly mad or just always pretended. He had merely raised the incriminating question about the revolver because you couldn’t put anything past someone so insidious and idle. In nowise did he wish to claim that the Werewolf had killed the apothecary; after all, it remained possible that his hideous fit of rage owed purely to indignation against the question or was even affected. But he shouldn’t neglect to direct the high authority’s attention to the dubious fact that on the day of the murder, the arrested had already been gone since noon and only reappeared before his front door the day after the funeral. If, then, after everything, the Werewolf proved guilty before the high court, said the mender—whereby the dome of his bosom inflated like a turkey’s before the also interrogated greengrocer—he should like most respectfully to recommend that he alone might lay claim to the offered reward for discovering the murderer. Meanwhile, the accused sat there perfectly stubborn-faced; only when his disappearance was mentioned did he become noticeably disquieted, and his compressed lips seemed again to be fighting back tears. Still, nothing more came of his interrogation than obstinate denial or the odd headshake at most, all while he furrowed his brow as though he couldn’t well comprehend the matter. And since in his wife’s testimony, she just swore blind that she had never sighted her husband with any such or other revolver, the fiercely eager newspaper readers had to appease their ardent desire for justice by immersing themselves in new scandalized editorials on the public’s insecurity in general, as well as on the sinister Werewolf and his having walked free for years in particular.

Meanwhile, progressing investigations revealed that around when the army employed revolvers of that oft-mentioned system, the accused had actually been a sergeant, and indeed for the horse artillery; also that at the hour of the murder, he had really not been in his home. But above all, the mender, who had in the meantime become considerably esteemed by the sympathetic citizenry and from day to day earned more of their favor, inquired around eagerly and succeeded in determining that for years, the arrested’s wife had incurred a sizable amount of small debts with every monger, merchant, butcher, baker, and craftsman in town and once loudly berated her husband before their neighbors for—in her own words—his measly pension and despicable indolence; moreover, on the day before the robbery, she had been engaged in cleaning for the apothecary’s family, so that she could have heard about the rail ride to the city beforehand and informed the Werewolf, thus leaving none more to doubt that he had intended with the bloody deed to supplement his paltry provisions, be that with or without his wife’s knowledge, and still kept the stolen banknotes hidden somewhere. Opinion was split only over whether he had come to the wicked decision from true lunacy or through continually calculating that steadily feigned derangement would leave all iniquity unpunished.

To all the well-meaning people’s great satisfaction, in the next trial, which was a public one, the latter opinion appeared to be confirmed; for when the arrested was confronted one by one with all the particulars of his suspicious lifestyle, it was clear to see how the sturdy man gradually stumbled, as it were, out of his habitual stiff-neckedness and finally threw a helpless look at the amiably smiling public prosecutor, who, upon unseverely returning the look—which back then was most astonishing to see from a prosecutor—asked the distressed defendant with rather a winsome voice if he wouldn’t finally unburden and cleanse his conscience with a brave confession unto God and man, whereat the Werewolf was overwhelmed with such a sob that most of the ladies in the gallery, even the apothecary’s widow, couldn’t help but loudly cry along. But all this took him so aback that for bewildered stammering he couldn’t reply with a single clear word but rather, tears rolling down his trembling beard, frantically choked out now a yes, now a no, now nodded with compunctious gestures, now shook his head defiantly. Nothing more could be brought out of him; and so, until his conscience ripened to full confession or other conclusive evidence of his guilt came to light, he had to be led back into custody.

Just when the populace already felt essentially consoled, albeit all the more essentially occupied with the unresolved question of whether the court might justifiably condemn the villain to death or merely lock him up for life in a madhouse, two almost incredibly contradictory yet police-verified newspaper reports set the people’s moral suspense a truly dreadful objective. The first report announced that early in the morning following the trial, the Werewolf hanged himself in his holding cell with a torn-off shirtsleeve strip and scrawled on the cell’s limestone walls the words: “I can’t go on. I don’t know anymore. Righteous Heaven, there is a God.” Whereas the second report said that that same morning, the public prosecutor received from the lawyer representing the apothecary’s widow a long express letter according to which the Werewolf was not the murderer but her husband was a suicide. And in fact the sorely tried lady knew this at first sight of the body, when the investigation officers showed her the cavalry revolver and at once she recognized it by a rust spot as belonging to the deceased from—as it was called then—his one-year voluntary military service. However—thus explained her lawyer—in order to honor her marital obligation and, as best she could, uphold the departed’s good name, both in the moral and especially the Christian sense, she had sought most self-abnegatorily to stay silent as long as possible and hence also accepted the sum insured without contradiction, particularly since her entitlement, as per the contract’s wording, could be considered unassailable. But because it seemed an innocent man would suffer for the bloody deed, and because the insurance company had also determined, regrettably, that the departed had squandered his assets in stock market speculations and so presumably had only staged the homicide to save his family from bankruptcy, his client felt she could no longer suppress the sad truth. She also maintained hope that, though her husband had committed a grave sin, common human compassion would nonetheless accept his frightful self-sacrifice as sufficient atonement and not punish his namesakes for it too—which hope was then indeed fulfilled most open-heartedly by both the friendly public prosecutor and the sentimental citizenry, especially when it was learned that the good-natured widow had come to an amicable agreement with the insurance company and privately repaid a third of the received sum.

Unfortunately for the hanged Werewolf, her confession had come delivered a few hours too late. But luckily it was sure to happen that the widows of the two suicides, since the second could have rightfully sued the first for compensation, had likewise to come privately to an amicable agreement. It also remained undetermined whether the Werewolf hadn’t perhaps harbored the criminal intent of murderous robbery that day he left his dwelling; and anyway there was a certain kind of higher justice to be found in the otherwise discomfiting fact that this state-supported malingerer, whose guilty conscience hadn’t even permitted him the peaceful enjoyment of his pension, had found justice straightforwardly by his own hand. Much more dismaying to the surprised populace’s educated sector was the tremendous postural ability that enabled the gentle, pious apothecary up to the last minute to prepare the appearance of murder-robbery and, still in death throes, cast both his revolver and timepiece out the railcar window. But most worrisome was the uncertainty, which for a long time provided every thorough newspaper reader with the richest conversational material about whether ultimately the Werewolf, as suggested by his mysterious wall inscription, had really gone insane and, following the friendly prosecutor, thought himself the murderer. To the enemies of civil order, naturally, that seemed an absolute certainty; indeed, one ruthless author of the time called it outright a governmental failure and an almost greater exemplar of mind-beguiling \textit{Suggestion}—as the cultivated Germans of the time expressed it—than that of the famous Captain of Köpenick.
\end{document}