\documentclass[12pt,a4paper]{article}
\usepackage[english]{babel}
\usepackage{microtype}
\usepackage[a4paper,margin=2in]{geometry}
\usepackage[T1]{fontenc}
\usepackage{fontspec}
\usepackage{ebgaramond}
\usepackage{dashrule}
\usepackage{verse}
\tolerance=855

\title{Fellowship}
\author{by Franz Kafka\\translated from the German by Justin Verceles}
\date{}

\begin{document}

\maketitle
We are five friends. We came out of a house once, one after the other; the first emerged and positioned himself by the gate; then the second emerged—or rather, glided out easily like a bead of quicksilver—and didn’t position himself far from the first; then the third, then the fourth, then the fifth. Finally we all stood in a row. Passers-by took notice and pointed at us, saying, "The five have emerged from the house." Since then we’ve lived together. It would be a peaceful life if not for a sixth constantly butting in. He doesn’t do anything to us, but we find him tiresome, and that’s doing enough. Why does he impose himself where he’s not wanted? True, we five didn’t always know each other, and you could say that even now we don’t know each other, but what’s possible and bearable between us five is the opposite with that sixth. Besides, we’re the five; we don’t want to be the six. And what’s the sense in this constantly being together, anyway? Even with us five, there’s no sense in it, but we’re together now and it will stay as such. But a new integration? We can’t have that, just on the grounds of our experiences. How should all that be explained to him? Giving him long explanations, we might as well be including him in our circle. We would rather explain nothing and not include him. However much he likes to pout his lips, we elbow him away, but however much we do that, he comes back. 
\end{document}