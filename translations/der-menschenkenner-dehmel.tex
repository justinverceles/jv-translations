\documentclass[12pt,a4paper]{article}
\usepackage[english]{babel}
\usepackage{microtype}
\usepackage[a4paper,margin=2in]{geometry}
\usepackage[T1]{fontenc}
\usepackage{fontspec}
\usepackage{ebgaramond}
\tolerance=840

\title{A Keen Judge of Men, and His Balance\\ \large{\textit{A novella from within a misanthrope}}}
\author{by Richard Dehmel\\translated from the German by Justin Verceles}
\date{}

\begin{document}

\maketitle

Jan Goderath was his name, and he was proud of his name. He’d restored honor to it when none trusted the old trading house anymore. And now he was passing through a foreign city, which suddenly brought to mind that time of suffering, and couldn’t interpret his gloomy mood; the whole city appeared sunken in sorrow.

Granted, a man had died: a man of the people, and an honest one too—even his enemies had to admit that. And he’d died before his time, fighting an agonizing laryngeal disease—a victim of his own eloquence. But what did Jan Goderath, a man of means, of the world, an independent chieftain of merchants, care about some  decrepit friend of the people! and an Italian moreover! Truly he abhorred these people. What did he share with a fool who had enthused in his suffering as did other fools! He was a Hamburger and a keen judge of men; how could the pouting faces of this foreign city’s rabble have affected him?

And Genoa especially—\textit{la superba}, as these sons of glorious fathers still called their marble townlet: what spirit had thus obsessed these bankrupt wretches? He surveyed the passersby; the brilliant morning light pleased him suddenly. Was this the same indolent, shamelessly garrulous crowd that had annoyed him just yesterday? Everyone walked as creepily as usual, almost even more creepily without their needless arm-waving, and none appeared indolent to him. As always, the narrow \textit{corso} seethed thickly with people’s heads, through which a wagon only rarely shoved its way; but the coachmen didn’t scream today, every voice sounded subdued, as though muffled by the gray palaces, and the faces seemed to adjust to the majestic walls, which abutted dismally in the blue sky. Even when a pretty woman passed, not one doggish pair of lustful black eyes followed her; in all these eyes smoldered a dreamlike solemnity—what was the meaning of that?!

Down by the harbor, he’d already noticed that today the work proceeded without any noise or cursing or laughter; even the muleteers in their quarries lashed their burdened beasts less roughly. Yet those—well, those were workingmen; a dead egalitarian might’ve really meant something to them. But here, in the inner city, what did these swindling salespeople, these layabouts and woman-worshippers, have to do with a man of the people! And what did all the foreigners here have to do with him! What gave the meager Frenchman there with the orange blossom in his buttonhole such a somber expression that both pillars of the portal he happened to be waiting before seemed like his natural frame, despite his stylish traveler’s hat? Did death do that?

No—such a penitent people was too light-hearted for that. Just last week in Pisa, he’d seen a high, widely admired official laid to rest; the entire city was on their feet, all bells sounded, eight barefooted monks bore the catafalque, all their brothers stepped forward as well as gold-trimmed purple priests, thereamong maidens in white dresses and children with green garlands in their hair, all with great burning candles, choir boys sang litanies, two Jesuit priests led the broken widow, the women of the procession weeped loudly—and an hour later there wasn’t a trace left of the whole street spectacle. And the Pisans held a reputation of thoroughness; he himself had felt comfortable with them—the populace’s blood must’ve been infused with that of Germanic conquerers centuries ago.

And today, here in Genoa, where normally every Romanic weed was put on display, there hung only silence since the early morning. He felt as though he walked among a flood of pilgrims. What had made everyone so strangely withdrawn? The dead folk hero hadn’t even been interred with pomp. No monk or priest had carried along his unadorned wooden coffin; six bareheaded workers had borne him, no tears had fallen, and no bells sounded. Or was it just that? Had this unusual silence impressed upon the praters’ souls? This colorless impression: the cortege of a hundred men dressed in black, all with bare heads, hats in hand, somber and wordless, walking pairwise behind the bier under the muggy blue sky. He’d even seen a navy officer lift his cap.

And didn’t Jan Goderath himself have to admit that it was these men’s forefathers who had once built this humble street of palaces, with the stringent force of its outer walls and its quiet interior temerity! He entered one of the mighty stairwells. If a smocked man had come now through this sweltering, stiffening peristyle, he would’ve doffed his hat before him. What was with him?! He couldn’t be brought out of balance by the impression of a few dozen mourners! His best years were behind him; he was already over thirty. Certainly, the impression had been beautiful, beautiful and earnest, maybe even noble. But that needn’t have disturbed his peace; earning it had been difficult enough. So what was he concerned with foreign misery! It couldn’t be helped, after all. What was he concerned with human suffering at all! As though there could be happiness without suffering. That would hold true till eternity.

He stepped back onto the street. And again he felt worked upon by the still glimmer of everyone’s eyes. Or was he only disturbed by the light reflected off the marble pavement? He walked over into the narrow streaks of shade; it was as though he were walking through a web that bound together all the dark heads. And yet none looked sad. There hung only a kind of devotion among them—as though they were listening for something distant, something clear. Death couldn’t do that, surely? Nor reverence? What did the deceased and his vague visions for the future mean to the fox-face over there, or those two professors? What was this constraining feeling that pervaded this whole city? and him as well! He’d already overcome entirely different moods that had affected him much more nearly: the time his brother poisoned himself—whose eyes were just as touching as these brown baselings here. Indeed, there was the time his father had died of a heart attack, and \textit{he} alone had saved everything.

He turned into the crowded space before the post office. The mood here was even stranger. The intense heat made everyone’s face still more agitated; this high agitation seemed to hover up to the building’s arcades. Even the disguised knife-dealer, whose creeping smile was otherwise as venal as his daggers, moved today like a banished prince in his blue-embroidered Dalmatian coat. Hardly a single clear word could be heard. Whenever someone spoke, they seemed to be contemplating something else, something forgotten, secret. What was with that? None of these sluggards here had loved the deceased! And \textit{he}, Jan Goderath senior: love—he could almost burst out laughing—that was a feeling he’d had quite enough of! His brother had broken him of that. He sighed heavily; what did he care about some laryngeally ill apostle of progress! about the whole running murmur! Perhaps if he were to close his eyes a little, the mood would pass. No, of course not; that only rendered its weight more oppressive—it seemed to him as though he were standing in his hometown, lost like a blind man, amid a crowd of churchgoers. He wouldn’t bear it any longer. Good thing the German painter was expecting him! The bust should be finished today; while posing, he was sure to find his balance again. He made toward the upper city.

For balance, indeed, was the highest thing—sound reason. It’d soothed him in his rage when he’d nearly smitten his brother, the dead scoundrel, the blackguard who had wished to make a deceiver of him! Yes, he was stronger than his love; he’d stood the test. Only how did he come upon interrogating his feeling today? Had the feeling been too weak when his reason was then so strong? That was no balance! Otherwise there would be harmony in his soul. He’d now traveled a year and believed to have overcome everything, and a few hundred whispering men could throw him? A herd with no comprehension of itself! He ran his fingers over his forehead. Now, thanks to art—he had to smile—soon he would be out of the din. Only occasionally did he find someone creeping here; they appeared like mere shadows; something seemed to beckon them down.

He climbed up the broad step street to the upper \textit{corso}. Already he felt the Apennine air despite the searing sun. It was a marvel of a city, all but equal to nature’s richness. What tremendous work was betokened by just the foundation walls it ascended all round the mountain terraces, by the hundreds of stone steps here, the surrounding walls’ ashlars zigzagged there around the \textit{corso}, by all the bridged chasms, and by the fortifying blockwork rowed hoarily atop the bare ridge crest up above—that was all the work of man! He recalled the inscription he’d read that morning down at the port, engraved into the palace that the Genoese people gifted an elderly Doria long ago: “\textit{ut, maximo labore jam fesso corde, otio digno quiesceret.}” He translated the bad Latin to himself: “so that he whose heart now wearies of immense work may rest in honorable leisure.” A shudder passed over him; all around here, on all these mountain slopes that half-encircled him, loomed the work of hundreds of thousands of men.

He turned and looked down at the city. How the lofty united with the base—palaces and streetways, the flat roofs and the towers, gardens, and vast masses of houses—in the surging, shining white of noon. There lay the villa Negro, with her gardens of laurels and myrtles, cypresses, palms, lemon trees, all Oriental flowers and every northern hardwood—never before had she appeared so lovely. He seemed to hear the splashing of her fountains, her grottoes’ little waterfalls, and, at her feet, the maze of ravine-like alleyways, in circles all around her, this spiderweb he’d just escaped. How the old and the new coalesced now under the blue, emblazoned heavenly vault! Every dark blemish, even the weathered church domes, appeared transfigured, down to the bustling harbor. How everything seemed to strive unto him, from alow and afar—mankind down below, the lighthouse, the ships, the dazzling, silvery white sea—he had to close his eyes.

A howling whistle tore them back open. In the leftward valley, a locomotive came steaming out the tunnel that ran bow-wise under the city; he reckoned he was standing directly over it. Were the earth to open up now, he would plummet into the shaft, the \textit{corso}’s walls passing over him. The work of thousands more, hidden! Perhaps including the men who today had borne the deceased. What if now those thousands of men wished to destroy their work? What would hinder them? A few dozen barrels of dynamite methodically distributed along the tunnel would send the city hurling into the harbor—fortress, jailhouse, madhouse and all. Already he could hear the teetering megaliths’ thunderous falls, the raging ocean, and the howling hurricanes. The palace roofs reared, steeples flew through the air, domes burst, and gardens danced. Marble statues shot into the seething sea in fervid yellow arcs, painting galleries went aflame, shipwreck, libraries. Through the darkened skies, through smoke and fire and clouds of debris resounded cries from citizens’ ruptured abdomens; and up above the revengement, on the smoke-wreathed heights of the Apennine Mountains stood those fiery-eyed thousands, bethinking them of the martyrs who had sacrificed themselves there—ready for a new future.

He wiped the sweat from his cheeks. What was with him! Was he already seeing spirits by daylight, like the village shepherds behind Hamburg? What was that compulsion? The men down below hadn’t appeared threatening but rather beseeching, as though they were looking to attain something. What did he have to do with that! He stretched himself. Yes, these strange seeking eyes—he nodded and walked farther; now he was nearly there. Remarkably, sometimes the painter too had these eyes: half begging, half demanding, the poor devil. Only now they were gray—gray as the North Sea, like his own, and yet like dogs’ eyes. Yes—like a bloodhound before a hunt: ravenous, skittish. And this slanted felon’s forehead! The matty brown goatee! The short legs! Really he found the fellow repugnant, and he suited this Romanic rabble: half lazzarone, half genius.

Why ever had he sought him? Why have himself painted by him? By this defiler of art! The way he would always stare at him, as though wishing to brush the soul from his body—and for nothing but rude patchwork. What had led him to this man?! Was it his being from Hamburg, his hometown? Pshaw—homesickness! How ridiculous! A disease of children! Or was it that he’d been friends with his brother? Well, maybe; he probably wished intentionally to assess himself. For it’d been one sunset two years ago when the three had stood together up behind Hamburg, out on the heights of the Elbe, at their feet the river’s prospect. It ran as broad as though the ocean itself began there. And the painter had turned away, staring at the smoking villages beyond which seemed to burn in the evening glow; for \textit{he}, Jan Goderath senior and successor, made a show of brotherly love—he’d believed the weakling again, they were flesh and blood after all—two days before he came to know it, came to despise it, this flesh and blood, this whole lot of humans. What did he care about this one! He’d probably known about everything, perhaps even helped falsify the transfer. Now—tomorrow he would keep on traveling, whether the painting was finished today or not.

So he entered the house. It was cool here, the stone steps freshly washed; now at once he would have peace. If only the painter could know how the mob had broken him. Yes—balance! Feeling and reason harmonized like two rulers of equal strength—if that could be painted, if that were present in a single visage, in one soul of earthly manhood, then they should be friends! There stood the goatee in the doorway. Slaveling! So he addressed him with the familiar \textit{du}—he gave him his hand—they approached the easel. He dried his forehead. “Didn’t you make the chin too prominent? I look like Napoleon before Moscow.” The goatee replied, grinning, “It wouldn’t be your only resemblance.” Oh, indeed! Was that supposed to flatter him? “I resemble none. And I certainly don’t claim that Corsican potbelly.” The painter, sheepishly: “The chin is good. Just wait for the eyes to finish; really it just comes down to the eyes.” “Oh? Well, then we’d better begin.” “Yes.”

He climbed up the footboard and leaned against the frame of posts. The squalid room was oppressively warm. A horn signaled from the Appenines. They looked silently into each other’s eyes; only the sound of the paintbrush was still audible. And how the man stared at him now! How he toiled for his crumbs, like the other hundreds of thousands! Was he pitying him—a pauper? He, the wealthy heir to Goderath? Ridiculous! He hadn’t even pitied his own brother when the latter had begged him fare to America. And now he should pity this total stranger? this utter incompetent? “Did you really love each other?” he suddenly heard as though from afar. What possessed this numbskull now? “Damn love!” he nearly screamed. Why then? something in him asked. “Pardon me!” he heard. Silence.

And again the eyes fixed upon him. And again as gray as the North Sea. And in the gray, something flickered. Just what was that? It was eerie. And it was many miles away—like a web between them, a glimmering stream, and therebeyond burning villages. And across the stream came thousands toward him, bareheaded, pairwise, carrying a body. And stared at him with their human eyes, ravenous, skittish, half begging, half demanding. As if there were something in him which they sought—something forgotten, distant, clear. And suddenly it beamed up within him and overflowed to them—a light, a sea, a glister amid the mist. “What’s with you, man?” called a voice—he staggered, stumbled, and lost his balance. And his hot tears blinded him, and blindly he staggered into two arms and kissed the beard that just a moment ago had repulsed him; he kissed it and cried like a child, and laughed, and then pulled himself together. Oh, it was more than reason and feeling! It wasn’t pity, no, it was love! Woeless o’er flesh and blood! Harmony and balance! It was that all-enlivening love.

His knees trembled, he had to sit down. He felt die prematurely that sickly workingman, for the love of the future; he felt live the longing of thousands to become like brothers, for the love of freedom; he felt all the victims of labor, for the love of all, and of the lives of all. And \textit{he}? He’d disdained men—he, Goderath, who should be their keenest judge! He extended his hands to the painter: “I’ve sinned against my brother . . .”
\end{document}