\documentclass[12pt,a4paper]{article}
\usepackage[english]{babel}
\usepackage{microtype}
\usepackage[a4paper,margin=2in]{geometry}
\usepackage[T1]{fontenc}
\usepackage{fontspec}
\usepackage{ebgaramond}
\usepackage{dashrule}
\tolerance=325

\title{A Race\\ \large{\textit{A visionary sketch}}}
\author{by Richard Dehmel\\translated from the German by Justin Verceles}
\date{}

\begin{document}

\maketitle

The race is about to begin. The lots have been drawn and the selectees now stand ready before the magician priest’s festively decorated wigwam.

I know I’m the priest in the tent. I lie frozen in a magical slumber and feel thousands of years old. In my dark chamber, I see all things shimmer with their inner light and direct in the soul every step and gesture made by my warriors outside. Already I want to move my warden’s hand to give the starting signal.

Thereupon the chieftain Kjugi lays his brown left hand’s red pointer finger upon the bright blue fist-sized sun painted on the skin over his heart, lays his yellow-starred right hand on his revolver, which flashes on his girdle’s hip-ring, and calls, “Let me run alone with the foreign warrior!”

And it happens. Soon the two are alone, among the endless hill-range.

The chieftain runs and runs. At times, the sun in heaven submerges under the gray mossy undulating sea of his track. He runs and listens and won’t look back till he hears the foreigner collapse. 

The hills begin to glow as pale as sand; only in the shadowed valleys does moss still nestle to the ground. The chieftain runs and listens so fervently that soon he’ll hear nothing more. The blue sun over his heart is nearly melted away; the yellow star on his right hand smears in sweat. The chieftain looks back.

The foreigner slowly climbs at his left over the next hill-ridge. He appears as light as a phantom; his gray-white headdress glistens silvery in the evening light. He stands and smiles.

The chieftain’s eyes well with fury. He gasps, ”What are you stopping for? Run, as you should, if you’re really as fast as me!” And his right hand feels his revolver.

The foreigner stands and glistens and smiles. “Run as you wish!” he replies in a monotone; “I can do nothing but stay behind you.” At once he looks almost corpse-like; his headdress glisters like a specter’s hair. Kjugi the chieftain laughs proudly.

He runs as he wishes; the foreigner stays behind him. The sun in heaven has gone down. The hills stretch endlessly into the pallor; even in the shadowed valleys nestles no more moss. The chieftain Kjugi wades in the sand; the foreigner remains ever behind him.

The chieftain Kjugi wades in sweat. The star on his right hand has long dissolved; the stars in heaven illumine the red that trickles down from the tip of his pointer finger to his left loin. Soon he’s as brown as when he was born. Again he looks back.

And again the foreigner stands at his left, in the dark of the next vale. His hair is as gray as the evening’s moss; but over his heart stares a faint, fist-sized, bright blue sun. He laughs mutely.

The chieftain Kjugi’s knees waver. He wheezes, “Who are you?” The foreigner laughs.

The chieftain yanks his revolver from its girdle. “Who are you?!” he cries into the darkness.

The foreigner stirs his right hand willfully. “I am the spirit of Kjugi”—so he breathes and raises it; the chieftain sees a yellow star.

He sees the pale blue sun stare at him—he sees a hand cock the revolver’s hammer—over the hills at his right, a second headdress emerges, wavering—a sudden bang flashes through the dark, and a cry—two gasping warriors collapse—the victor has shot himself in the heart.

And I awake—who are you, dream-spirit?
\end{document}