\documentclass[12pt,a4paper]{article}
\usepackage[english]{babel}
\usepackage{microtype}
\usepackage[a4paper,margin=2in]{geometry}
\usepackage[T1]{fontenc}
\usepackage{fontspec}
\usepackage{ebgaramond}
\usepackage{dashrule}
\tolerance=2160

\title{The Yellow Cat\\ \large{\textit{A burlesque}}}
\author{by Richard Dehmel\\translated from the German by Justin Verceles}
\date{}

\begin{document}

\maketitle

Nothing impresses a more definitive effect than the undefined. With its application, my brother Ernst used to recount to me his experiences. Now he’s dead. Shortly before his end, he wrote me the following.

When the woman for whom I wanted to leave my own would talk about her husband, she always appeared ugly to me. Her brownish skin would yellow, her wild hair seemed blacker and lower in her forehead, her eyes’ pitchy glister became piercing, and her mouth’s zestful expression became helpless. I called it her maid-face; but it was inexplicable to me.

She commanded the man; but that, she could no longer enfetter. His body had become unbearable to her, his derisive wit no less. His revengefulness she feared not, and his good nature she despised. She raved for freedom like a Russian princess. So why did she stay with him?

Granted, she had a child with him. But she didn’t like touching them, though she thought she loved them very much. She preferred playing with my daughter and desired to bear my son.

She didn’t even rely on him for money; he wouldn’t have withheld it from her, he was an honorable man. That he could shoot me dead in a duel, she feared just as little; I wouldn’t have ventured my life in his honor—(here my brother Ernst is lying)—and I needn’t have done it for her sake, my existence meant more to her than people’s judgment.

“Is it because you’re ashamed of what your parents might think?” I asked her one day while we were on an outing together.

“Maybe”—she smiled childishly; her thousand freckles shimmered. Then she made her snake-face as though trying to swallow her words; and shortly afterward she laughed like a Bacchante.

We strolled through my favorite village, formerly a village of the crown of Frederick the Great. It was Good Friday. On Easter she wanted to visit her homeland; spring on the Rhine was paradise to her. When she spoke about it, she appeared to me like the Virgin Mary in the flesh; her brown, night-like eyes were transfigured.

The chestnut buds now grew thick and green; some already extended a finger. The maple blossoms glistened golden yellow through the blue evening. “I’ll make myself a fairy’s scepter from that,” she said, “when my father and I ride through the mountains.”

I looked at her—“There are wicked fairies too, you”—and wanted to kiss her. A twitchy, skew little wrinkle emerged between her black brows, as it did whenever she felt superior to me. Her plump nose twitched too. I kept my kiss.

Suddenly her pupils grew large and covetous. “Look, how eerie!” she whispered and pointed across the street. All her freckles seemed to disappear, even from her lips. Her swelling mouth grew darker. That was her witch-face—the sixth of hers that I discerned.

I went over with her. On an artificial hill stood a strange little house behind the fence. It was always unoccupied, I knew that. In the bright gloaming, it looked even spookier.

Two giant plane trees stretched their still-bare branches like a corpse’s bones over the flat roof. The walls were pale and spotty. To the left, a crooked arborvitae swayed its gloomy leaves. From the middle of the front wall raised a round spired little tower that reminded me of Chinese hats; the door was locked. Around the small arched windows crawled braids of Gothic flourishes; the glass was as black as my companion’s pupils. Between the house’s right corner and the plane’s trunk set the yellow-red sun.

“I’d like to stay here now and then,” said the pretty woman. That moment, a big yellow-red cat came slowly over the hill’s rear, just as though out of the sun, and sat down before the locked door.

The scene upset me, so deeply atmospheric it was. The creature’s black-brown eyes vaguely reminded me of a child-murdering woman from a wax museum. The sun was gone; the fur now looked even yellower, almost silken. It stared down at us and blinked; I shivered. I clapped my hands; it ran away. 

The pretty woman had started and looked at me somewhat reluctantly. “I don’t like house cats,” I said harshly. She nodded silently and took my arm with devotion. We turned to go home, but the sinister impression wouldn’t leave me. The more tenderly she spoke to me, the more upset I became. I blamed it on the holiday. Over and over, amid our whispers, I kept hearing Jesus’ consolation to the crucified murderer: today shalt thou be with me in Paradise.

I kissed her goodbye, almost sheepishly, and said with a laugh, “Farewell, Magdalene.” She made her virgin-face.

That night I dreamed—(my brother Ernst considered dreams experiences, too)—that I was looking out the window and saw the strange little house standing opposite me and to the side. Starlight glimmered in the black windowpanes. Suddenly they grew blindingly bright. The whole house was illuminated up into the hole-riddled chimney. The windows and door opened up; and from every opening, from every hole and hatch, down from the roof and walls, leapt countless black cats and scattered to the four winds. Finally, one big reddish-yellow cat came slowly out the door, stared after me, blinking, and likewise disappeared into the darkness. Then the house shut itself up just as soundlessly and darkened again at one blow.

Morning came. I sat with my wife over coffee; we discussed our separation. “If you feel certain,” she said in her faithful voice, “that the other is better suited for your happiness than me, I won’t stop you”—the bell rang in the corridor.

The maidservant reported that there was a young woman who wished to speak with me; I went into the next room. A large young lady met me; I frightened. She was dressed all in red and yellow silk, the straw hat covering her black hair had a spray of fake maple blossoms; she had all the pretty woman’s same traits, only not so Saracenic—tamer, as it were. I stood speechless.

Maybe it was her after all? No! She’d gone away yesterday. And every one of her facial features was foreign to me. And she didn’t have a sister.

The lady smiled childishly; her thousand freckles shimmered. “You don’t recognize me,” she discerned quietly; I shook my head uneasily. “I’m the yellow cat,” she said drolly; I shivered. Then it occurred to me; maybe it was a prank from the pretty woman—she had friends in acting circles. The lady blinked, and a skew little wrinkle emerged between her black brows; “I’m meant to collect you,” she whispered.

A snakish look flashed from her eyes and bewitched me. “Now?” I asked. “Now!” We went.

We descended the steps silently; a wagon was waiting before the door. We rode through a number of streets, still in silence; she didn’t seem to notice me at all. The streets grew narrower, the houses ever higher, and the environment unfamiliar to me. At one point she gave a brief nod; I saw a black cat scurry through a gateway. At another, she smoothed her tangled hair with her yellow glove. Finally the wagon stopped; I followed her submissively.

We went through a dank court, then up multiple iron stairs, and through many dim passages. It was a veritable labyrinth of a house; the air smelled musty. She halted before a pitch-black hallway door and pressed something invisible. The door flew open, I stood dazzled. I was confronted by a piercing splendor, like a thousand candelabra.

When I came to my senses, I was standing in a seemingly immeasurable hall; before me, behind me, on all sides were mirrored walls. And through the middle of the hall, lengthways, reflected from all sides, stood an endless row of black-clothed ladies, turning soundlessly, and mouse-gray gentlemen, hopping soundlessly, as to the rhythm of some extrasensory dance music.

None of the women—(I gather hence that my brother Ernst was still dreaming)—had just one man—most had two, some even three; others appeared to have a dozen, if the mirrors didn’t deceive me. All of them, merrily though they turned, exhibited an oddly helpless sort of gloom, almost like automata; the middlemost one held a crying child in her arm. 

Whenever one of the women inflected a little lower to one of the men, he gave an especially high hop, so that his mouse-gray coattails, which otherwise flapped down on the floor, would wag through the air. Then the other gentlemen, especially the fat ones, threw him angry looks; but the lady would smile childishly, then even fattest men softened again.

I began to feel dizzy; I looked around for my yellow conductress. A shudder crept over me—all her freckles were gone. Her pupils were haggishly large, and she stood there like the princess of this ballet and shook her Bacchanalian locks. Her hair had come undone, her straw hat lay on the ground. In her right hand, she held the fake maple spray and swung it like a scepter. Her face was dark brown, her zestful nose looked twisted. She nodded to me.

Just then, the mirrored door behind her flew open anew; and silently through it, in a mouse-gray frock, coattails between his fingertips, came the pretty woman’s husband hopping straight toward me. I nearly burst out laughing, when at once, in the slowly reclosing mirror-door, I saw myself, horrified, in the mouse-gray frock, and suddenly I began to hop along.

I fight desperately to stop. I throw the pretty woman my most earnest looks. All in vain. The deeper into my eyes she blinks, the higher I hop.

I try to come nearer to the husband. I try to goad him into seizing me. He looks at me mockingly and hops. I try to prove to him—I hop. I try to show him—he hops. I try to dash him to the ground—we hop.

I want to fall to the pretty woman’s feet. I want to beg her mercy. I try and try and can’t—her brown skin becomes an ugly yellow, her hair looks ruffled like an animal’s mane, and lower over her forehead, her regard turns piercing, her fulsome mouth’s expression turns helpless: she dons her maid-face.

I scream out in pain—and awake.

At my bedside stood my wife with our daughterling and raked my hair. “Father,” said the little one thoughtfully, “you looked so terribly funny in your sleep.” I kissed both their hands.

Since that morning—so my brother Ernst closed his queer missive—the yellow cat never intimidated me again. Soon afterward he died in a duel; he’d wanted to bid the lady farewell, and the walls had ears. He died by the husband’s trembling hand—he, the admirable marksman. Nothing impresses more definitively than the undefined.
\end{document}